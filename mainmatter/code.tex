\chapter{Code Analysis}\label{ch:code}

The code itself is an index of the differences between the \gls{os} and \software{\FONTsmallcaps{0mq}} implementations. The latter provides more abstract \glspl{api}, more consistent and organic functionality, an easily expandable core, and built-in scalability; the former is compatible with common non-\software{\FONTsmallcaps{0mq}} sockets, and has some minor advantages in low level socket options.

Due to the broad and subjective nature of some of these differences, the following analysis is mostly qualitative.

\section{Abstraction}\label{sc:code:abstraction}

\software{\FONTsmallcaps{0mq}}'s more abstract \glspl{api} allow for shorter, more concise, and overall clearer code. The total amount of lines dedicated to the \gls{netcode} go from \num{253} with \gls{os} sockets, to \num{151} with \software{\FONTsmallcaps{0mq}} sockets: a \SI{40,64}{\percent} decrease. Specifically, the larger decreases are in the methods that open the connections: \code{open_RX_connection()} has \SI{80,88}{\percent} less lines; \code{open_TX_connection()} has \SI{82,86}{\percent} less lines. Other than the obvious advantages in readability, less code lines result in less maintenance, less side effects when altering the code, and less failure points.

\begin{table}[!ht]
	\centering
    \begin{tabular}{lccc}
    	& \FONTsmallcaps{os} sockets & \software{\FONTsmallcaps{0mq}} sockets & difference \\
		\TABLEmidruler
		\file{UDPConnector.cpp}     & \num{253} & \num{151} & -- \SI{40,64}{\percent} \\
		\code{UDPConnector()}       & \num{3}   & \num{7}   & + \SI{57,14}{\percent} \\
		\code{open_RX_connection()} & \num{68}  & \num{13}  & -- \SI{80,88}{\percent} \\
		\code{open_TX_connection()} & \num{70}  & \num{12}  & -- \SI{82,86}{\percent} \\
		\code{receive_packet()}     & \num{19}  & \num{17}  & -- \SI{10,53}{\percent} \\
		\code{send_packet()}        & \num{19}  & \num{17}  & -- \SI{10,53}{\percent} \\
        \TABLEbottomruler
    \end{tabular}

	\caption{Code Analysis: \gls{netcode} lines}\label{tb:code:lines}
	\CAPTIONadd{Some quantitave measures on the amount of lines in some of the \gls{netcode} code. Negative difference is in favour of \software{\FONTsmallcaps{0mq}} sockets, positive difference is in favour of \FONTsmallcaps{os} sockets.}
\end{table}


As an example, the two following snippets both set up a socket before a \code{bind()}:

\begin{codelist}{UDPConnector.cpp --- open\_TX\_connection() --- \FONTsmallcaps{os} sockets}{144}
    if (m_socketTXfd > -1)
        return false;

    struct sockaddr_in sock;
    uint32_t address;

    int opt;
    int result;

    InetPton(AF_INET, ip, &address);

    m_socketTXfd = socket(AF_INET, SOCK_DGRAM, 0);

    if (m_socketTXfd == -1) {
        return false;
    }

    opt = 1;
#ifdef _WIN32
    result = setsockopt(m_socketTXfd, SOL_SOCKET, SO_REUSEADDR, (char*) &opt, sizeof(opt));
#else
    int result = setsockopt(m_socketTXfd, SOL_SOCKET, SO_REUSEADDR, &opt, sizeof(opt));
#endif  // _WIN32

    if (result == -1) {
        return false;
    }

    opt = 1;
#ifdef _WIN32
    result = setsockopt(m_socketTXfd, SOL_SOCKET, SO_BROADCAST, (char*) &opt, sizeof(opt));
#else
    result = setsockopt(m_socketTXfd, SOL_SOCKET, SO_BROADCAST, &opt, sizeof(opt));
#endif  // _WIN32

    if (result == -1) {
        return false;
    }

    memset(&sock, 0, sizeof(sock));
    sock.sin_family      = AF_INET;
    sock.sin_addr.s_addr = address;
    sock.sin_port        = htons(port);
\end{codelist}

\begin{codelist}{UDPConnector.cpp --- open\_TX\_connection() --- \software{\FONTsmallcaps{0mq}} sockets}{100}
	std::string addr = std::string(ZMQBRIDGE_REMOTE_PROTOCOL) + "://" + std::string(ip) + ":" + std::to_string(port);
\end{codelist}

\FLOATnoindent \gls{os} sockets require defining a \code{struct sockaddr_in}, to enclose some basic protocol options (\code{SOCK_DGRAM}, \code{SO_REUSEADDR}), the socket's \gls{ip} address and port, and additional meta settings; \software{\FONTsmallcaps{0mq}} sockets achieve the same result through a single string, with meta settings internally set by \software{\FONTsmallcaps{0mq}} itself: e.g.\ \code{"tcp://192.168.1.100:5555"} is formed by:

\begin{itemize}
	\item \FONTbold{\code{"tcp"}} --- the protocol (line \num{155} in the \gls{os} sockets implementation);
	\item \FONTbold{\code{"192.168.1.1"}} --- the \gls{ip} (lines \num{153} and \num{185} in the \gls{os} sockets implementation);
	\item \FONTbold{\code{"5555"}} --- the port (line \num{186} in the \gls{os} sockets implementation).
\end{itemize}

\FLOATnoindent Also of note is memory handling, necessary for the  \code{struct sockaddr_in} (a \code{memset()} call), and entirely missing in the \software{\FONTsmallcaps{0mq}} solution.

Since \software{\FONTsmallcaps{0mq}} sockets are proper objects (of type \code{void*}) and not simple \code{int}(s) representing file descriptors, the constructor's code is instead marginally more involved:

\begin{codelist}{UDPConnector.cpp --- UDPConnector() --- \FONTsmallcaps{os} sockets}{60}
UDPConnector::UDPConnector() {
	m_context  = zmq_ctx_new();
	m_socketRX = zmq_socket(m_context, ZMQ_SUB);
	m_socketTX = zmq_socket(m_context, ZMQ_PUB);

	m_payloadRX = (char*) malloc(UDPCONNECTOR_ME_PAYLOAD_BUFFER);
	m_payloadTX = (char*) malloc(UDPCONNECTOR_ME_PAYLOAD_BUFFER);
}
\end{codelist}

\begin{codelist}{UDPConnector.cpp --- UDPConnector() --- \software{\FONTsmallcaps{0mq}} sockets}{63}
UDPConnector::UDPConnector() {
	m_payloadRX = (char*) malloc(UDPCONNECTOR_ME_PAYLOAD_BUFFER);
	m_payloadTX = (char*) malloc(UDPCONNECTOR_ME_PAYLOAD_BUFFER);
}
\end{codelist} % FORMATTING fit this in the page

\FLOATnoindent \code{m_context} is required by \software{\FONTsmallcaps{0mq}} sockets, since it encapsulates state information and some general settings (here left to their defaults): at creation, sockets are assigned to a specific context.

\section{Ease}\label{sc:code:ease}

\software{\FONTsmallcaps{0mq}} provides bindings that allow on-the-fly changes on some aspects of the netcode. The used protocol can be changed by simply altering the protocol part of the \code{addr} variable (\code{ZMQBRIDGE_REMOTE_PROTOCOL}); the sockets paradigm can similarly be changed in the constructor (\FONTsmallcaps{pub}/\FONTsmallcaps{sub}) to any other allowed combination.

For example, switching from an \gls{tcp} \FONTsmallcaps{pub}/\FONTsmallcaps{sub} paradigm to an \code{inproc} (in-memory) \FONTsmallcaps{pair} paradigm only requires changes to three lines of code:

\begin{codelist}{\gls{tcp}}{1}
#define ZMQBRIDGE_REMOTE_PROTOCOL "tcp"
m_socketRX = zmq_socket(m_context, ZMQ_SUB);
m_socketTX = zmq_socket(m_context, ZMQ_PUB);
\end{codelist}

\begin{codelist}{inproc}{1}
#define ZMQBRIDGE_REMOTE_PROTOCOL "inproc"
m_socketRX = zmq_socket(m_context, ZMQ_PAIR);
m_socketTX = zmq_socket(m_context, ZMQ_PAIR);
\end{codelist}

\FLOATnoindent Achieving a similar results with an \gls{os}-based implementation would require an extensive rewrite of the low level code.

\section{Expandability}\label{sc:code:expandability}

\software{\FONTsmallcaps{0mq}} is by construction fully scalable. Depending on the socket paradigm used, this may allow other software to be freely added to the system, with no changes required to the \gls{netcode} layer: different \glspl{ts} could be simultaneously used by adding additional socket, simply by instantiating new \code{zmq_socket()}; a \gls{ts} could subdivide data using the \FONTsmallcaps{pub}/\FONTsmallcaps{sub} paradigm's subscription lists; and more.

\section{Recap}\label{sc:code:recap}

This chapter has been devoted to the analysis of the code style difference between the implementations. \software{\FONTsmallcaps{0mq}} sockets have a large advantage thanks to higher abstraction (\fref{sc:code:abstraction}), coding ease (\fref{sc:code:ease}), and easier expandability (\fref{sc:code:expandability}).
