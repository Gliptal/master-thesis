\chapter{Performance Analysis}\label{ch:performance}

\begin{keywords}
	me, vtd
\end{keywords}

The two following chapters are devoted to analysing the differences in performance and coding between the \gls{os} sockets and \software{0mq} sockets approach. The goal is to provide a consistent and valid set of data, that can be used to empower proper decision making, by weighting the pros and cons of each of the two approaches. As such measures are repeated multiple times in identical conditions, to obtain datasets with proper confidence levels.

Hardware, graphics settings, and scene are identical for all of the performance tests: the simulation is ran on the test 8-track, with \num{1} player, \num{2} \gls{ai} vehicles controlled by \gls{vtd}, a variable \gls{me} \gls{sendrate} (dependent on the \gls{framerate}), and a \SI{120}{\hertz} \gls{vtd} \gls{sendrate}. See \fref{tb:performance:hardware} for the hardware specifications.

\begin{table}[!ht]
	\centering
    \begin{tabular}{lllll}
    	\TABLEmulticolumn{1}{c}{\acrshort{os}} & \TABLEmulticolumn{1}{c}{\acrshort{cpu}} & \TABLEmulticolumn{1}{c}{\acrshort{ram}} & \TABLEmulticolumn{1}{c}{disk drive} & \TABLEmulticolumn{1}{c}{\acrshort{gpu}} \\
		\TABLEmidruler
        Windows 10 Home & Intel i7-6820HK & \SI{32}{\giga\byte} \SI{2.4}{\giga\hertz} & \SI{1}{\tera\byte} \acrshort{hdd} & GeForce GTX 1080 (\SI{8}{\giga\byte}) \\
        \TABLEbottomruler
    \end{tabular}

    \caption{testing hardware}\label{tb:performance:specifications}
\end{table}


\section{Framerate}\label{sc:performance:framerate}

\begin{definition}{framerate}
\end{definition}

\begin{table}[!ht]
	\centering
    \begin{tabular}{l|l}
        hardware & Intel i7-6820HK, 32GB 2.4GHz, 1TB HDD, GeForce GTX 1080 (8GB) \\
		source & Fraps 3.5.99 (benchmark) \\
		conditions & 20 runs (30s), 1 player + 2 AIs, variable / 120Hz \\
		confidence & 0,999 \\
    \end{tabular}

	\caption{Performance Analysis: framerate testing environment}\label{tb:performance:framerate}
\end{table}


\gls{me}'s \gls{framerate} is analysed, since the user's visual experience depends on it. Higher \glspl{framerate} are obviously better: the minimum acceptable \gls{framerate} is around \SI{30}{\fps}, with \SI{60}{\fps} or more being optimal on the more common \SI{60}{\hertz} monitors. %FIXME units with fps

A game engine's \gls{framerate} depends on a large number of conditions, mainly graphics settings, how busy the scene is, underlying \gls{os} usage, hardware availability, and network issues. Generally speaking, better frames are achieved by:

\begin{itemize}
	\item \FONTbold{low graphics settings} --- Lower rendering resolution ($1080p$ or less), low quality/amount of \gls{fx}, basic shaders, no reflections, reduced \gls{pp} effects like \gls{dof}, blur, bloom, no \gls{aa} techniques.
	\item \FONTbold{sparse scenes} --- Scenes with less and/or more simple objects, like buildings, vehicles, or drivers.
	\item \FONTbold{idle \gls{os}} --- None or limited \gls{os} processes running in the background and utilizing the available hardware.
	\item \FONTbold{high-end hardware} --- High-clock \glspl{cpu}\footnote{Note that for game engines higher single core clock is usually more important than high thread counts.}, high-end gaming \glspl{gpu}, large (\SI{32}{\giga\byte} or more) amounts of fast \gls{ram}, \glspl{ssd} as opposed to \glspl{hdd}.
	\item \FONTbold{stable network} --- Low latency, no or minimal packet loss.
\end{itemize}

Measures were obtained with the benchmark feature of \software{Fraps} $3.5.99$. For each configuration (\gls{os} sockets, \software{0mq} sockets, unmodified \gls{me}) the measures were repeated \num{20} times (\SI{30}{\second} per run), obtaining three quantities:

\begin{itemize}
	\item \FONTbold{minimum} --- The minimum \gls{fps} reached during the \SI{30}{\second} run.
	\item \FONTbold{maximum} --- The maximum \gls{fps} reached during the \SI{30}{\second} run.
	\item \FONTbold{average} --- The average \gls{fps} of the \SI{30}{\second} run.
\end{itemize}

\FLOATnoindent Each of these $3\times3=9$ values was then averaged, and a \num{0,999} confidence interval calculated with a Student t's distribution. See \fref{ax:math:statistics} for some hints about statistical analysis.

\begin{image}
	{performance/framerate}{0.6}
	{framerate results}
	{im:performance:framerate}
	{}
\end{image} % TODO finish collecting FPS data

The tests show no appreciable difference in performance between \gls{os} sockets and \software{0mq} sockets. Maximum is \SI{143}{\fps} for both, with only a higher variability in the \software{0mq} case ($\pm$\SI{1,8}{\fps} as opposed to \SI{0,38}{\fps}). Minimum shows again a much greater variability for \software{0mq} ($\pm$\SI{6,7}{\fps}): with \SI{99,9}{\percent} confidence it can be assumed the minimums \gls{fps} are the same. Average \gls{framerate} displays more interesting results: \software{0mq} sockets are better performing by a very small margin, between \SI{0,10}{\fps} and \SI{7,90}{\fps} taking into account the confidence intervals.

In conclusion, \software{0mq} sockets do provide slightly better average performance than \gls{os} sockets; minimum and maximum \gls{fps} are not affected.

\section{Network}\label{sc:performance:network}

\begin{definition}{netcode}
\end{definition}

\begin{table}[!ht]
	\centering
    \begin{tabular}{l|l}
        hardware & Windows 10 Home, Intel i7-6820HK, 32GB 2.4GHz, 1TB HDD, GeForce GTX 1080 (8GB) \\
		source & Wireshark 2.6.1 (statistics) \\
		conditions & 20 runs (30s), 1 player + 2 AIs, variable / 120Hz \\
		confidence & 0,999 \\
    \end{tabular}

    \caption{network testing environment}\label{tb:performance:network}
\end{table}


Data about network performance (the \gls{netcode}) is also collected. In a realistic working environment, \gls{me} and the \gls{ts} would most probably reside on a high-speed, dedicated wired connection. As such, performance measures like ping and packet loss are not collected: both depend heavily on the network infrastructure, so any results would not be easily transferable to real life situations.

Collection and basic analysis were done with the statistics feature of \software{Wireshark} $2.6.1$. As with \fref{sc:performance:framerate} for each configuration (\gls{os} sockets, \software{0mq} sockets, unmodified \gls{me}) the measures were repeated \num{20} times (\SI{30}{\second} per run),to obtain three quantities:

\begin{itemize}
	\item \FONTbold{packets per second} --- The average number of packets that each second insists on the channel (i.e.\ sent and received by both \gls{me} and \gls{vtd}).
	\item \FONTbold{load per second} --- The average amount of data that each second insists on the channel (i.e.\ sent and received by both \gls{me} and \gls{vtd}).
	\item \FONTbold{total load} --- The total amount of data exchanged between \gls{me} and \gls{vtd} in the whole duration of the run (\SI{30}{\second}).
\end{itemize}

\FLOATnoindent Each of these $3\times2=6$ values was then averaged, and a \num{0,999} confidence interval calculated with a Student t's distribution. See \fref{ax:math:statistics} for some hints about statistical analysis.

\begin{image}
	{performance/networkfrequency}{0.6}
	{network results (packet frequency)}
	{im:performance:networkfrequency}
	{}
\end{image}

\begin{image}
	{performance/networkdata}{0.6}
	{network results (data frequency)}
	{im:performance:networkdata}
	{}
\end{image}

\begin{image}
	{performance/networkload}{0.6}
	{network results (total data)}
	{im:performance:networkload}
	{}
\end{image}

As expected, total load and data frequency are identical, when taking into account the confidence intervals. This is because the bulk of the packets is formed by the payload: the data is relatively heavy due to the lack of encoding, hence the protocol's overhead bytes are mostly irrelevant. Packet frequency is instead completely different, due to the different protocols used. \gls{os} sockets with \gls{udp} only send about \SI{341}{\packets\per\second}, whilst \software{0mq} sockets with \gls{tcp} send twice that amount: \gls{tcp} (and in part \software{0mq}'s overhead) require small acknowledgement packets to be exchanged during the communication.\footnote{Note that all the results heavily depend on the sendrates: trivially, the higher, the more packets and data are exchanged.}

To conclude the network analysis, \gls{os} sockets only have an advantage in the packets frequency. Given the dedicated nature of the network infrastructure that is usually utilized in these projects, this advantage is hardly relevant: \software{0mq} thus shows it has no negative impact on network performance.

\section{Hardware}\label{sc:performance:hardware}

\begin{table}[!ht]
	\centering
    \begin{tabular}{lllll}
    	\TABLEmulticolumn{1}{c}{\acrshort{os}} & \TABLEmulticolumn{1}{c}{\acrshort{cpu}} & \TABLEmulticolumn{1}{c}{\acrshort{ram}} & \TABLEmulticolumn{1}{c}{disk drive} & \TABLEmulticolumn{1}{c}{\acrshort{gpu}} \\
		\TABLEmidruler
        Windows 10 Home & Intel i7-6820HK & \SI{32}{\giga\byte} \SI{2.4}{\giga\hertz} & \SI{1}{\tera\byte} \acrshort{hdd} & GeForce GTX 1080 (\SI{8}{\giga\byte}) \\
        \TABLEbottomruler
    \end{tabular}

    \caption{testing hardware}\label{tb:performance:hardware}
\end{table}


Hardware performance is also collected, to analyse which between \gls{os} or \software{0mq} sockets has the higher impact on \gls{cpu} and \gls{ram} usage. The same values were also taken with the unmodified \gls{me}, to obtain a baseline and see if the \gls{middleware} has any impact on resource usage.

Data was obtained with Windows' built-in performance monitor features. To ensure as best as possible independent subsequent runs, both \gls{me} and \gls{vtd} were restarted each time; additionally, instead of absolute \gls{ram} values relative deltas are used. Four quantities per \gls{middleware} configuration are measured:

\begin{itemize}
	\item \FONTbold{minimum \gls{cpu}} --- The minimum overall \gls{cpu} usage reached during the \SI{20}{\second} run.
	\item \FONTbold{maximum \gls{cpu}} --- The maximum overall \gls{cpu} usage reached during the \SI{20}{\second} run.
	\item \FONTbold{average \gls{cpu}} --- The average overall \gls{cpu} usage of the \SI{20}{\second} run.
	\item \FONTbold{\gls{ram} delta} --- The difference in used \gls{ram} between the moment the run starts and the moment it ends.
\end{itemize}

\FLOATnoindent Each of these $4\times3=12$ values was then averaged, and a \num{0,950} confidence interval calculated with a Student t's distribution. See \fref{ax:math:statistics} for some hints about statistical analysis.

\begin{image}
	{performance/hardwarecpu}{0.6}
	{hardware results (\gls{cpu} usage)}
	{im:performance:hardwarecpu}
	{}
\end{image}

\begin{image}
	{performance/hardwareram}{0.6}
	{hardware results (\gls{ram} usage)}
	{im:performance:hardwareram}
	{}
\end{image}

Again, taking into account the confidence intervals, the \gls{middleware} with the \gls{os} sockets implementation has no impact on four out of five quantities: only the average \gls{cpu} usage shows a minimal impact. Interestingly, the \software{0mq} sockets implementation has a relevant effect on \gls{cpu} usage: between an \SI{4,8}{\percent} and \SI{19}{\percent} increase in maximum load; between an \SI{6,5}{\percent} and \SI{12}{\percent} increase in minimum load; between an \SI{5,4}{\percent} and \SI{13}{\percent} increase in average load. This is probably due to the additional overhead introduced by \software{0mq}'s high level functions, and the additional packets required by the \gls{tcp} protocol and \software{0mq}'s \FONTsmallcaps{pub/sub} model. \gls{ram} delta variability is too high to appreciate any changes between the implementations.

When hardware usage is concerned, \gls{os} sockets have a significant advantage over \software{0mq}, with lower all-around hardware usage. It must be noted though that the higher hardware usage does not translate in lower \gls{me} performance (see \fref{sc:performance:framerate}): hypothetically, this should hold true until \gls{me}'e performance is not \gls{cpu} bound.

\section{Proxy}\label{sc:performance:proxy}

\section{Results}\label{sc:performance:results}

\section{Recap}\label{sc:performance:recap}
