\chapter*{Introduction}\label{ch:introduction}
\TOCadd{toc}{chapter}{Introduction}

The concept of "simulation" is becoming increasingly relevant in all the branches of ground-based transport: whether developing and testing sensors, using deep learning methodologies for \gls{ad}, or analysing driver behaviour in arbitrary traffic situations; simulating the required process allows for cheaper, faster, and less rigid production loops. Since both the academic and the enterprise research environments push for similar long-time goals (barring some expected differences), a tool tailor-made for one can, with the appropriate changes, be fruitfully used by the other. As such, collaborations between universities such as \gls{unitn} and software companies such as \company{ioTech} may prove remarkably successful, by promoting an healthy exchange of workforce, technologies, and know-how.

\gls{unitn} has been looking into either developing, co-developing, or downright acquiring a fully modular driving simulator, to replace the software currently used in a number of different research projects. To that end, an analysis was made on some of the competitors in the simulators market: by properly weighing the \glspl{feature} specifically requested informed decisions may be taken for future collaborations. This analysis and its results are presented in \pref{pr:motivation}, \cref{ch:feasibility}.

A related opportunity arose for a collaboration with the newly formed \company{ioTech}, a subsidiary of \gls{sms}. This company the developer, owner, and distributor of \software{Project Cars}, a consumer-grade driving simulator and one of the leading videogames in the genre, thanks to the team's ten-plus years experience, a focus on realistic physics, and state-of-the-art graphics. \company{ioTech}'s aim is to branch the videogame's underlying engine in the commercial and enterprise scene, providing high-fidelity visuals (unmatched by the current enterprise solutions) on a modular and extensible simulation core.

An internship period of six months was agreed upon with \company{ioTech}: the goal was to develop an agnostic, plug-and-play, and network-based integration of a (any) third-party vehicle traffic system into \software{Project Cars}' game engine. More specifically, the state and actions taken by the agents in a traffic system were to be displayed to the user in the game engine; the user was to drive a car in that game engine, interacting in real time with those same traffic agents. As a proof-of-concept, the traffic software produced by \company{Vires Simulationstechnologie GmbH}, a fully featured enterprise solution to simulate realistic car traffic flows in arbitrary road topologies, was successfully plugged into the system. The work produced during this collaboration is detailed in \pref{pr:development}: learning and familiarizing with the various software involved (\fref{ch:software}), designing and coding a \gls{middleware} (\fref{ch:middleware}), and successfully integrating it into the game engine (\fref{ch:integration}).

Whereas the initial implementation of the \gls{middleware} leveraged low-level raw \gls{os} functions for its network sockets (specifically \gls{udp}), a second implementation (unrelated from the internship) was also developed, that utilizes the existing \gls{middleware} framework \software{0mq}. Details on this framework, its implementation, and the key differences can also be found in \fref{ch:middleware}.

Finally, a performance-driven analysis was done on the \gls{middleware}, concentrating on two main aspects: its impact on overall system performance (based on a number of metrics, some related to the game engine), and code development concerns. The two socket solutions (\gls{os} vs.\ \software{0mq}) were compared: ultimately \software{0mq}, with only a small drop in performance, allows for greater code modularity and expansion, easier maintainability, and less complex code requirements; still, other considerations apply, showing that both approaches are equally viable. \pref{pr:analysis} is entirely devoted to this analysis.
