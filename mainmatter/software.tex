\chapter{Software Descriptions}\label{ch:software}

\begin{keywords}
	me, middleware, vtd
\end{keywords}

This chapter describes the software involved in the development. \gls{me} is the game engine on top of which \software{Project Cars} is built; \gls{vtd} by \company{Vires Simulationstechnologie GmbH} contains the traffic software, that sends and accepts data about its entities in the proprietary \gls{rdb} format; the \gls{middleware} is a home-grown solution to connect the two previous software.

Additionally, the two socket libraries methods used to implement two distinct versions of the \gls{middleware} are also summarily presented. More information and details on their implementation can be found in \fref{ch:middleware}.

\section{Overall Structure}\label{ch_software:overall}

Before analysing in detail each software an overall view is presented, treating each as a black box with a specific function. The \gls{middleware} offers two functions (or a two-way function): it converts data from a proprietary format (like \gls{rdb}) to \gls{me}'s accepted format, and the opposite. Details on how it accomplishes this, how it abstracts to any possible proprietary format, and the intermediate third \gls{sae} format can be found in \fref{ch:middleware}.

\begin{table}[!h]
    \begin{tabularx}{\FLOATtextwidth}{lXX}
    	software & input & output \\
		\TABLEmidruler{}
        \gls{me}         & data on traffic entitles (\gls{me} format) & data on player vehicle (\gls{me} format) \\
        \gls{vtd}        & data on player entity (\gls{rdb} format) & data on traffic entities (\gls{rdb} format) \\
        \gls{middleware} & \MAKECELLcell[l]{data in \gls{rdb} format \\ data in \gls{me} format} & \MAKECELLcell[l]{data in \gls{me} format \\ data in \gls{rdb} format} \\
        \TABLEbottomruler{}
    \end{tabularx}

    \caption{software functions}\label{tb:software:functions}
\end{table}


The overall black box structure is summarized as in \fref{im:software:structure}. The difference in implementation between \gls{os} sockets and \software{0mq} sockets is only visible after grey boxing the \gls{middleware} and \gls{vtd}, so it is omitted here.

\begin{figure}[!ht]
	\FLOATcenter{}
	\FIGUREinclude[width=\FLOATtextwidth]{mainmatter/images/software/structure.pdf}

    \caption{overall structure}\label{im:software:structure}
\end{figure}

Data about the traffic entities flows from \gls{vtd} to the \gls{middleware} in the \gls{rdb} format; their respective \code{in} and \code{out} ports are network sockets, with the two host \glspl{os} being reachable via \gls{ip} with each other. The data is converted into an \gls{me} digestible format by the \gls{middleware}, and then supplied to \gls{me}: this is actually done in-memory (the implementation of the \gls{middleware} runs within \gls{me}), hence the ports are in this case abstractions for memory management. Similarly, data about the player vehicle flows in the opposite way: \gls{me} supplies the \gls{middleware} with the data, the \gls{middleware} converts it into the \gls{rdb} format, and sends it to \gls{vtd}.

This feedback loop is repeated continuously for the entire duration of the simulation. The right-to-left flow moves the "ego" car within the traffic simulation, allowing its entities to react in real time; the left-to-right flow allows \gls{me} to render the traffic entities on screen, displaying them to the end-user. Time and spatial synchronization issues are omitted in this simplified description, details can be found in \fref{sc:integration:approach}. % TODO: check correct ref

This structure is akin to a client-server paradigm: here the client is \gls{me}, and the server is \gls{vtd}\footnote{This distinction is purely arbitrary, since it's a 1:1 communication loop}. \Glspl{sendrate} and \glspl{tickrate} are not synchronized: both software are running at their own simulation paces; \gls{vtd}'s sendrate is fully configurable; \gls{me}'s sendrate is dependent on its framerate.

\section{Madness Engine}\label{sc:software:madnessengine}

\section{Virtual Test Drive}\label{sc:software:vtd}

\subsection{Runtime Data Bus}

\section{Middleware}\label{sc:software:middleware}

\subsection{\textsc{os} Sockets}

\subsection{0mq Sockets}

\section{Recap}\label{sc:software:recap}
