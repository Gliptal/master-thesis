\chapter{State of the Art}\label{ch:stateoftheart} % CONTENT how can I connect this directly to the thesis? maybe it's better to just keep it incidental

\begin{keywords}
	attribute, easiness, feasibility, feature, framework, practicability
\end{keywords}

This chapter describes the analysis performed on behalf of \gls{unitn} on some of the main competitors in the driving simulator market. Informed decision may then be made with regards to which simulation \gls{framework} might best suit the ongoing research projects. The work on the \gls{middleware} stems from the implementation of one feature of the ones presented ("traffic"), specifically developed on \software{Project Cars}, but realistically with any other \gls{framework} in mind; as such, this chapter is only tangentially related with the focus of the thesis, and mainly serves as an introduction on how the requirements for the \gls{middleware} came to be.

\fref{sc:stateoftheart:features} lists and describes the set of \glspl{feature} required by \gls{unitn}. \fref{sc:stateoftheart:frameworks} lists and describes a number of existing simulation \glspl{framework}, analysing the main advantages and shortcomings of each. \fref{sc:stateoftheart:comparison} compares the various \glspl{framework} by scoring two \glspl{attribute} for each \gls{feature}, providing an overall weighted score used to numerically rank the different \glspl{framework}.

The following analysis has been written without taking into account the information and experience collected during the collaboration with \company{ioTech}.

\section{Features}\label{sc:stateoftheart:features}

\begin{definition}{feature}
\end{definition}

In the scope of state of the art analysis, detailing specific \glspl{feature} eases the comparison between the different \glspl{framework}. Note that there is no requirement that any of these \glspl{feature} be developed from the ground-up, and/or in-house: a valid alternative might be to integrate pre-existing solutions.

\Glspl{feature} are separated based on their domain, either software or hardware; in each section, the \glspl{feature} are listed in order of perceived importance. An associated weight $w$ is also given to every \gls{feature}\footnote{$w \in \MATHSrationals: 0 < w \leqslant 1$} (see \cref{tb:stateoftheart:weights}), as to numerically quantify the relative importance of each: this weight is then used when comparing the various \glspl{framework} in \fref{sc:stateoftheart:comparison}. The values are arbitrarily chosen in accordance with \gls{unitn} requirements, based on their current and planned research projects.

\begin{table}[!ht]
    \begin{tabularx}{\FLOATtextwidth}{*{10}{Y}}
    	\TABULARXmulticolumn{5}{c}{software} & \TABULARXmulticolumn{2}{c}{input hardware} & \TABULARXmulticolumn{3}{c}{output hardware} \\
        \TABULARXpartialruler(lr){1-5} \TABULARXpartialruler(lr){6-7} \TABULARXpartialruler(lr){8-10}
        model & map & world & traffic & sensor & \FONTsmallcaps{ff} & touch & \FONTsmallcaps{vr} & wheel & seat \\
		\TABLEmidruler
        \XCOLORcellcolor{color5}\(1.0\) & \XCOLORcellcolor{color4}\(0.8\) & \XCOLORcellcolor{color3}\(0.6\) & \XCOLORcellcolor{color2}\(0.4\) & \XCOLORcellcolor{color2}\(0.4\) & \XCOLORcellcolor{color5}\(1.0\) & \XCOLORcellcolor{color3}\(0.6\) & \XCOLORcellcolor{color5}\(1.0\) & \XCOLORcellcolor{color3}\(0.6\) & \XCOLORcellcolor{color2}\(0.4\) \\
        \TABLEbottomruler
    \end{tabularx}

	\caption{State of the Art: \glspl{feature} weights}\label{tb:stateoftheart:weights}
	\CAPTIONadd{Relative weights for the different analyzed \glspl{feature}, with \num{1} being most important and \num{0} being least important. Changing these weights alters the overall scoring in \fref{sc:stateoftheart:comparison}, and reflects different priorities in finding the ideal \gls{framework}.}
\end{table}


\subsection{Software}

\subsubsection{Model}

The simulator must be capable of accepting a mathematical model as an input, that must then be used by the game engine to properly simulate the behaviour of the car. Any changes made on this model should reflect in the simulation without requiring any modifications of the source-code (plug-and-play paradigm).

The purpose of this \gls{feature} is to allow developing, altering, or fine tuning the behaviour of a car via physical considerations alone. Separating the mathematical description from the coding implementation allows direct alterations to the behaviour of the simulator, without requiring any level of knowledge of the underlying code. Different models of cars, tire conditions, track conditions, and more may then be tested with the simulator without explicit additions or alterations.

\subsubsection{Map}

The simulator must be capable of recreating arbitrary street maps, or racetracks. The data used to create the map may come from different sources.

\begin{itemize}
	\item \FONTsmallcaps{3d} laser scanning is a process that uses laser-based technologies to convert real-life surfaces to clouds of virtual points. The resulting \FONTsmallcaps{3d} point cloud can then be modified and rendered with millimetric precision.~\cite{feasibility:laserscanning} This technique is usually used to faithfully recreate small unchanging scenarios like racetracks, thanks to the high-precision, highly automated, time-saving approach.
    \item Manually building the scenario with an editor grants total control on its characteristics since the presence, properties, and placement of every element can be freely fine tuned. This is the easiest data source to implement, although faithfully recreating real-life locations is time-consuming, and devising practical non-existing locations requires significant design skills.
    \item Data about a geographic location (such as roads, buildings, trails, train tracks) can be collected in independent bundles of structured, parsable data. By mapping each entity of the bundle to an asset of the engine the bundle can be programmatically analysed, and the entire location recreated by the engine. Any arbitrary road grid can thus be generated automatically, although the precision and soundness of the final result directly depends on the quality of the mapping algorithm and of the original bundle.
\end{itemize}

\subsubsection{World}

The simulator must be capable of rendering and handling world objects, both static and mobile; these objects should interact with the simulated car as realistically as possible, by altering the behaviour of the car (e.g.\ puddles, curbs), or by interacting with simulated smart sensors (e.g.\ pedestrians, road lines). A distinction should be made between generic world objects, and specific traffic related entities, to ease the implementation of the traffic \gls{feature}.

Mobile entities may be controlled by either simple or non-trivial \gls{ai} systems, depending on their role within the simulation.

\subsubsection{Traffic}

\gls{ai} traffic must be simulated with an acceptable degree of precision. Although traffic entities are technically world objects, the added complexity governing their behaviour warrants a specific \gls{feature}. The \gls{ai} should be formed by a core capable of providing basic behaviour, and an extendible architecture to allow future expansions such as:

\begin{itemize}
	\item entities following different rules based on their type (e.g.\ emergency vehicles);
    \item entities behaving with a degree of unpredictability;
    \item entities altering their behaviour based on the output of simulated sensors;
    \item entities dynamically changing their path based on an overall set goal.
\end{itemize}

Any smart sensors should be able to properly interact with the moving traffic, and ideally with similar simulated sensors on the \gls{ai} cars.

\subsubsection{Sensors}

Smart sensors may alter the behaviour of the simulated car (by providing data to its self-drive systems), provide the driver with information about the surrounding environment, or give feedback on the car's state. Generally, two paradigms can be used to simulate smart sensors.

\begin{itemize}
	\item Use engine \glspl{api} (e.g.\ raycasting) to directly collect the data handled by the sensor: this allows an easier development process, but creates an unrealistic simulation due to the reliance on functions not available to the real sensor.
    \item Fully recreate the physical phenomena (e.g.\ reflections) that the real sensor uses to collect its data: this requires in-depth development of both the sensor and the associated physics, but ultimately provides an highly realistic simulation of its inner workings.
\end{itemize}

Properly simulating smart sensors is a massive undertaking, and existing solutions use remarkably complicated technologies to achieve results. As such, integrating rather than developing may be the only realistic way of providing this \gls{feature}.

\subsection{Input Hardware}

\subsubsection{\FONTsmallcaps{ff}}

\gls{ff} effects must be felt on the steering wheel and the pedals. This feedback reflects the forces acting on the car at any given time, and not the force transmitted by the car's smart systems.\footnote{In a real-life car there exist no \gls{ff} motor in the steering wheel; the forces felt are rather of purely mechanical origin.}

\gls{ff} requires both a software component (that calculates the feedback forces based on the current forces acting on the simulated car), and an hardware component (a motor delivering the force to the steering wheel and the pedals). More advanced software and hardware components allow enhanced mappings, by e.g.\ providing individual feedback based on the status of each wheel.

\subsubsection{Touch}

The virtual \FONTsmallcaps{3d} space should be aware of the position of the hands and fingers of the real-life driver, ideally by camera recognition or \gls{vr} gloves. The advantages are twofold: systems can be handled with hand gestures without requiring additional input systems (e.g.\ keyboard, mouse); the driver has a visual representation of the position of his hands even whilst wearing a \gls{vr} headset.

\subsection{Output Hardware}

\subsubsection{\FONTsmallcaps{vr}}

The simulation must be compatible with \gls{vr} headsets. Full integration requires a number of \gls{vr}-specific options such as head positioning, scaling, supersampling, \FONTsmallcaps{3d} cockpits, eye distance, and more. \gls{vr} enhances the experience primarily by providing depth perception and complete immersion (currently at the cost of narrow \gls{fov}), limiting the requirement for multiple screens or expensive wide-angle projector setups.

\subsubsection{Wheel}

The steering wheel must be able to provide steering or counter-steering force based on any input given by the car's smart system. These forces are separate from the typical simulation \gls{ff}, since they do not represent the physical state of the car (what in real-life is controlled by mechanics alone), but rather represent the output of some smart system (a deliberate software-controlled motor output on the steering wheel).

For simulating reasons the steering wheel already has a motor used for \gls{ff} purposes: that same motor may also be used by the sensors. Having two separate dedicated motors would enhance the simulation, by allowing directly counteracting forces to be felt by the driver.

\subsubsection{Seat}

\gls{ff} effects must be felt on the simulator's setup seating. This feedback should reflect the forces acting on the car at any given time, in a similar manner to \gls{ff}.

A number of technologies make this possible, normally cantered around a motor that acts on one or more moving elements, matching as close as possible the forces applied to the car with forces felt by the real-life driver.

\begin{itemize}
	\item A moving platform with an immobile seat allows limited 2-axis or 3-axis movement: simulated g-forces are reproduced using gravity.
    \item A moving seat allows less limited 3-axis movement that a moving platform: similarly, gravity is leveraged to reproduce g-forces.
	\item Inflatable sacks can be worn or placed on the seat: g-forces are then reproduced by filling the sacks with air, providing pressure on the body of the driver.~\cite{feasibility:gametrix}
	\item Specific signals may be translated into matching low frequency sounds, that then drive a weighted motor: this may give remarkably precise and distinct rumbling effects.~\cite{feasibility:buttkicker}
\end{itemize}

\section{Frameworks}\label{sc:stateoftheart:frameworks}

\begin{definition}{framework}
\end{definition}

The simulation software listed here can all be considered \glspl{framework} due to their general capability of being modified (at different depth levels) by the user, and are analysed as such. These are a mix of consumer-grade, enterprise, and open source projects that share a common focus (real or potential) on driving simulation.

\subsubsection{OpenDS}

\begin{excerpt}{feasibility:opends}
    The flexible open source driving simulation.
\end{excerpt}

\software{OpenDS} is an open-source driving simulation, modifiable under the \FONTsmallcaps{gnu} general public license. The core architecture provides a basic rendering engine (capable of low-quality graphics), positional audio support, and notably a \planguage{Java} fork of the \software{Bullet} real-time physics engine~\cite{feasibility:bullet}: the latter provides \enquote{mesh-accurate collision shapes, experience of acceleration, friction, torque, gravity and centrifugal forces}{\cite{feasibility:opendsfeatures}}. Other relevant features include:

\begin{itemize}
    \item automated scenery generation;
    \item a \FONTsmallcaps{3d} editor;
	\item scenery-aware \gls{ai} cars and pedestrians;
    \item a traffic lights system;
    \item real-time data export;
    \item \software{Oculus Rift} \gls{vr} headset support;
    \item motion chair support.
\end{itemize}

\software{OpenDS} also provides high-level analysis and scenery creation tools, geared towards evaluating the performance of a driver in an arbitrary driving scenario. It thus lacks an accurate simulation of both the behaviour of the car, and the physical interactions it may have with the surrounding environment.

Six different purchase tiers are available, ranging from a \SI{99}{\$} "Student" option to a \SI{3150}{\$} "Pro Complete \FONTsmallcaps{ce}" option: notably, only the latter pricier tier includes the automated scenery generation tools. As such, the analysis of this \gls{framework} in \fref{sc:stateoftheart:comparison} is based on the "Pro Complete \FONTsmallcaps{ce}" tier.

\subsubsection{VDrift}

\begin{excerpt}{feasibility:vdrift}
    VDrift is a cross-platform, open source driving simulation made with drift racing in mind.
\end{excerpt}

\software{VDrift} is centred around simulation-grade driving physics. Unlike \software{OpenDS} the focus is on racing and proper physics simulation, with little to no effort put into systems and analysis. Features of interest are limited to:

\begin{itemize}
	\item numerous tracks and vehicles;
    \item limited racing \gls{ai};
    \item limited \gls{ff} support.
\end{itemize}

Additionally, the last release dates back to \printdate{2014-10-20}, with slow development times, and a small number of contributors.

\subsubsection{Unity}

\begin{excerpt}{feasibility:unity}
    The world's favourite game engine.
\end{excerpt}

\software{Unity} is a powerful and widespread game engine, capable of rendering \FONTsmallcaps{2d} and \FONTsmallcaps{3d} environments alike; a host of modules allows complete integration with existing technologies such as \gls{vr}.

Being a mature and fully customizable game engine, the main advantage is the possibility of implementing every \gls{feature} without having to rely on the support of \nth{3}-party \glspl{api}. The lack of high-level \glspl{api} specific to driving simulations though forces a longer and more involved development process, since most \glspl{feature} must be coded from the ground up.

Three monthly subscription tiers are available, ranging from a free "Personal" option to a \SI{125}{\$} "Pro" option; the latter allows negotiating source code access, and obtaining advanced support. Further analysis of this \gls{framework} is based on the "Plus" tier.

\subsubsection{Panthera}

\begin{excerpt}{feasibility:panthera}
	Panthera uses high-end physics and an excellent rendering engine. It contains controllers for motion platforms, steering feedback, pedals, dashboard, audio etc.\ as well as a scripting engine to define and customize the simulation.
\end{excerpt}

\software{Panthera} is a full-stack, modular software suite by \company{Cruden}; initially tailored for in-house \gls{hil} and \gls{sil} solutions, it has been since ported as a stand-alone desktop simulator. It \enquote{uses high-end physics and \CHARomissis\ It contains controllers for motion platforms, steering feedback, pedals, dashboard, audio etc.\ as well as a scripting engine to define and customize the simulation.}{\cite{feasibility:panthera}} More specifically it includes:

\begin{itemize}
	\item a generic interface that allows custom built simulation models;
    \item interfaces for scenarios and traffic;
    \item extensive output hardware customization;
    \item a dedicated tire simulation;
    \item \gls{hil} and \gls{sil} support.
\end{itemize}

Visuals are unmatched by any other professional engineering simulation available. It is unclear whether \gls{vr} support is available or not: a reference is made about \enquote{\glspl{hmd} [being] a factor in our next Panthera Free simulator software upgrade}{\cite{feasibility:pantheravr}}, but no mention can be found in the features lists; it is apparent though that \gls{vr} can be supported by the rendering engine.

No pricing listings are publicly available; additionally the proposed free software suite lacks, among other features, the \gls{ff} module and multiple monitor or projection support.

\subsubsection{Project Cars}

\begin{excerpt}{feasibility:projectcars}
    Project CARS 2 is the second installment in the best-selling and critically-acclaimed Project CARS motorsport franchise developed by Slightly Mad Studios.
\end{excerpt}

\software{Project Cars} is a consumer-grade driving simulation geared towards racing and hyper-realistic car handling. Being a direct port of a consumer-grade software, both visual and audio rendering are state of the art: as such, the immersion potential is the highest of all analysed \glspl{framework}, especially when considering \gls{vr} support.

Although the current available engine is closed-source and not modifiable, the software company \gls{sms} is in the process of developing a professional and enterprise version of their simulation and rendering engine. All details regarding functionalities, \glspl{api}, and cost are unknown at the time: all further analysis is thus done based on the capabilities of the current consumer product.

\software{Project Cars 2} has full support for most external peripherals, including \gls{vr}, \gls{ff} steering wheels, and moving seats. A strong focus is given to tire-simulation, driven by a technology named \software{Over the Limit}: the developers were helped by both \company{Pirelli} and a number of real-life drivers.~\cite{feasibility:projectcarstires} \gls{ai} and available tracks are limited to racing only, with no pre-existing support to city driving.

\subsubsection{SCANeR}

\begin{excerpt}{feasibility:scaner}
SCANeR studio is a complete software tool meeting all the challenges of driving simulation \CHARomissis\ it is a genuine evolving simulation platform, extendable and open, answering the needs of researchers and engineers.
\end{excerpt}

\software{SCANeR} is a modular simulation \gls{framework} that provides dedicated \glspl{gui} to create, alter, or configure its different parts; visual and audio rendering are basic, following a "function before form" paradigm. Relevant \glspl{feature} are:

\begin{itemize}
	\item dynamic traffic and pedestrians;
    \item tunable vehicle model;
    \item \planguage{XML}-based scenario creation.
\end{itemize}

Available information of the full set of capabilities is limited; the strong focus on the \gls{gui} of the sales pitch though seems to imply limited expandability out of the scope of what the original developers intended. Development might be forced to occur not on the source code of the software itself, but rather on a limited set of \glspl{api} or on separate parsers and configurators.

No pricing is publicly displayed.

\subsubsection{STISIM Drive}

\begin{excerpt}{feasibility:stisimdrive}
Whether for occupational therapy assessment and rehabilitation, driver training, or driver research, the versatile STISIM Drive brings you a new level of flexibility, ease and analytical insight.
\end{excerpt}

\software{STISIM Drive} is, similarly to \software{OpenDS}, a driving simulation focused on driver training, research and assessment, and scenario conception:

The possibilities of meaningfully extending the \gls{framework} seem limited by the scenario-centred list~\cite{feasibility:stisimdrivefeatures} of capabilities, that contains no mentions to peripherals, physics simulation and interaction, on-board systems, or smart sensors:

\begin{itemize}
	\item proprietary scenario definition language;
    \item modular programming;
    \item realistic roadway environments;
    \item data collection and analysis.
\end{itemize}

Minor importance is clearly given to the realism of the physical simulation, the quality of the graphics and audio, and the integration with input and output peripherals.

No pricing is publicly listed, although numerous pre-built solutions are proposed by the developing company, in the form of bundles of software and hardware.

\section{Comparison}\label{sc:stateoftheart:comparison}

\subsection{Attributes}

For each \gls{feature} two \glspl{attribute} are scored, in order to objectively assess the capabilities of a \gls{framework}:

\begin{definition}{practicability}
\end{definition}

\begin{definition}{easiness}
\end{definition}

For example, a framework with a large number of \glspl{api} but limited access to the source code would have an high \gls{easiness} score but a low \gls{practicability} score.

\subsection{Scoring}

Scoring is based on both objective and subjective observations on the capabilities and tools of the analysed \gls{framework}. A score between \num{1} and \num{5} is given to the \gls{practicability} ($s_p$) and \gls{easiness} ($s_e$) \gls{attribute} of each \gls{feature}: \cref{tb:stateoftheart:scoring} lists these scores.

\begin{table}[!ht]
    \begin{tabularx}{\FLOATtextwidth}{l*{10}{Y}}
        & \TABULARXmulticolumn{2}{c}{model} & \TABULARXmulticolumn{2}{c}{map} & \TABULARXmulticolumn{2}{c}{world} & \TABULARXmulticolumn{2}{c}{traffic} & \TABULARXmulticolumn{2}{c}{sensors} \\
        \TABULARXpartialruler(lr){2-3} \TABULARXpartialruler(lr){4-5} \TABULARXpartialruler(lr){6-7} \TABULARXpartialruler(lr){8-9} \TABULARXpartialruler(lr){10-11}
        & $s_p$ & $s_e$ & $s_p$ & $s_e$ & $s_p$ & $s_e$ & $s_p$ & $s_e$ & $s_p$ & $s_e$ \\
        \TABLEmidruler
        \software{OpenDS}       & \TABLEcellnum{2} & \TABLEcellnum{1} & \TABLEcellnum{4} & \TABLEcellnum{4} & \TABLEcellnum{4} & \TABLEcellnum{4} & \TABLEcellnum{5} & \TABLEcellnum{4} & \TABLEcellnum{3} & \TABLEcellnum{2} \\
        \software{VDrift}       & \TABLEcellnum{2} & \TABLEcellnum{1} & \TABLEcellnum{3} & \TABLEcellnum{2} & \TABLEcellnum{3} & \TABLEcellnum{3} & \TABLEcellnum{2} & \TABLEcellnum{2} & \TABLEcellnum{2} & \TABLEcellnum{1} \\
        \software{Unity}        & \TABLEcellnum{5} & \TABLEcellnum{1} & \TABLEcellnum{5} & \TABLEcellnum{2} & \TABLEcellnum{5} & \TABLEcellnum{3} & \TABLEcellnum{5} & \TABLEcellnum{2} & \TABLEcellnum{5} & \TABLEcellnum{1} \\
        \software{Panthera}     & \TABLEcellnum{5} & \TABLEcellnum{3} & \TABLEcellnum{5} & \TABLEcellnum{3} & \TABLEcellnum{4} & \TABLEcellnum{3} & \TABLEcellnum{5} & \TABLEcellnum{3} & \TABLEcellnum{4} & \TABLEcellnum{2} \\
        \software{Project Cars} & \TABLEcellnum{3} & \TABLEcellnum{2} & \TABLEcellnum{3} & \TABLEcellnum{2} & \TABLEcellnum{4} & \TABLEcellnum{4} & \TABLEcellnum{3} & \TABLEcellnum{2} & \TABLEcellnum{4} & \TABLEcellnum{2} \\
        \software{SCANeR}       & \TABLEcellnum{2} & \TABLEcellnum{1} & \TABLEcellnum{4} & \TABLEcellnum{4} & \TABLEcellnum{4} & \TABLEcellnum{4} & \TABLEcellnum{5} & \TABLEcellnum{5} & \TABLEcellnum{2} & \TABLEcellnum{2} \\
        \software{STISIM Drive} & \TABLEcellnum{2} & \TABLEcellnum{1} & \TABLEcellnum{3} & \TABLEcellnum{3} & \TABLEcellnum{3} & \TABLEcellnum{3} & \TABLEcellnum{5} & \TABLEcellnum{4} & \TABLEcellnum{2} & \TABLEcellnum{2} \\
        \TABLEbottomruler
    \end{tabularx}

    \FLOATskipbig

    \begin{tabularx}{\FLOATtextwidth}{l*{10}{Y}}
        & \TABULARXmulticolumn{2}{c}{\FONTsmallcaps{ff}} & \TABULARXmulticolumn{2}{c}{touch} & \TABULARXmulticolumn{2}{c}{\FONTsmallcaps{vr}} & \TABULARXmulticolumn{2}{c}{wheel} & \TABULARXmulticolumn{2}{c}{seat} \\
        \TABULARXpartialruler(lr){2-3} \TABULARXpartialruler(lr){4-5} \TABULARXpartialruler(lr){6-7} \TABULARXpartialruler(lr){8-9} \TABULARXpartialruler(lr){10-11}
        & $s_p$ & $s_e$ & $s_p$ & $s_e$ & $s_p$ & $s_e$ & $s_p$ & $s_e$ & $s_p$ & $s_e$ \\
        \TABLEmidruler
        \software{OpenDS}       & \TABLEcellnum{3} & \TABLEcellnum{3} & \TABLEcellnum{2} & \TABLEcellnum{2} & \TABLEcellnum{5} & \TABLEcellnum{5} & \TABLEcellnum{3} & \TABLEcellnum{2} & \TABLEcellnum{5} & \TABLEcellnum{4} \\
        \software{VDrift}       & \TABLEcellnum{3} & \TABLEcellnum{3} & \TABLEcellnum{2} & \TABLEcellnum{2} & \TABLEcellnum{2} & \TABLEcellnum{1} & \TABLEcellnum{2} & \TABLEcellnum{2} & \TABLEcellnum{2} & \TABLEcellnum{2} \\
        \software{Unity}        & \TABLEcellnum{5} & \TABLEcellnum{3} & \TABLEcellnum{5} & \TABLEcellnum{2} & \TABLEcellnum{5} & \TABLEcellnum{4} & \TABLEcellnum{5} & \TABLEcellnum{3} & \TABLEcellnum{5} & \TABLEcellnum{3} \\
        \software{Panthera}     & \TABLEcellnum{5} & \TABLEcellnum{4} & \TABLEcellnum{4} & \TABLEcellnum{2} & \TABLEcellnum{3} & \TABLEcellnum{3} & \TABLEcellnum{4} & \TABLEcellnum{3} & \TABLEcellnum{5} & \TABLEcellnum{4} \\
        \software{Project Cars} & \TABLEcellnum{5} & \TABLEcellnum{4} & \TABLEcellnum{4} & \TABLEcellnum{2} & \TABLEcellnum{5} & \TABLEcellnum{5} & \TABLEcellnum{5} & \TABLEcellnum{4} & \TABLEcellnum{5} & \TABLEcellnum{4} \\
        \software{SCANeR}       & \TABLEcellnum{3} & \TABLEcellnum{2} & \TABLEcellnum{2} & \TABLEcellnum{2} & \TABLEcellnum{1} & \TABLEcellnum{1} & \TABLEcellnum{2} & \TABLEcellnum{2} & \TABLEcellnum{2} & \TABLEcellnum{2} \\
        \software{STISIM Drive} & \TABLEcellnum{3} & \TABLEcellnum{3} & \TABLEcellnum{2} & \TABLEcellnum{2} & \TABLEcellnum{2} & \TABLEcellnum{2} & \TABLEcellnum{2} & \TABLEcellnum{2} & \TABLEcellnum{3} & \TABLEcellnum{2} \\
        \TABLEbottomruler
    \end{tabularx}

	\caption{State of the Art: \glspl{framework} scoring}\label{tb:stateoftheart:scoring}
	\CAPTIONadd{The scores presented here are somewhat arbitrary, and come from a subjective analysis on the limited information available about each \gls{framework}. They do not reflect objective measures on their \glspl{api} or code, and as such are only a rough indication of \gls{practicability} and \gls{easiness}.}
\end{table}


\subsection{Analysis}

To aid in the ordering of \glspl{framework}, a third \gls{attribute} ($s_f$) is used:

\begin{definition}{feasibility}
\end{definition}

Weighing allows importance to be shifted between \gls{practicability} and \gls{easiness}.\footnote{See \fref{ax:math:feasibility} for the mathematical description of the function.} Additionally, an overall score ($s$) is given to each \gls{framework} by applying a weighted mean to the \gls{feasibility} values\footnote{See \fref{ax:math:mean} for the mathematical description of the function.}: see \fref{sc:stateoftheart:features} for the weights of the \glspl{feature}.

Three different rankings are provided, each in order of overall score (from the best to the worst):

\begin{itemize}
	\item in \cref{tb:stateoftheart:equal} equal weight is given to \gls{practicability} (\num{0,5}) and \gls{easiness} (\num{0,5});
    \item in \cref{tb:stateoftheart:practicability} larger importance is given to \gls{practicability} (\num{0,7}) than \gls{easiness} (\num{0,3}), highlighting those \glspl{framework} that would best provide a complete implementation of the simulator;
    \item in \cref{tb:stateoftheart:easiness} larger importance is instead given to \gls{easiness} (\num{0,7}) than \gls{practicability} (\num{0,3}), highlighting those \glspl{framework} that would best provide easier to use tools and faster development times.
\end{itemize}

\begin{table}[!p]
    \begin{tabularx}{\FLOATtextwidth}{lY|*{10}{Y}}
        & \FONTbold{tot} & mod. & map & wor. & tra. & sen. & \FONTsmallcaps{ff} & tou. & \FONTsmallcaps{vr} & whe. & sea. \\
        \TABULARXpartialruler(lr){2-2} \TABULARXpartialruler(lr){3-3} \TABULARXpartialruler(lr){4-4} \TABULARXpartialruler(lr){5-5} \TABULARXpartialruler(lr){6-6} \TABULARXpartialruler(lr){7-7} \TABULARXpartialruler(lr){8-8} \TABULARXpartialruler(lr){9-9} \TABULARXpartialruler(lr){10-10} \TABULARXpartialruler(lr){11-11} \TABULARXpartialruler(lr){12-12}
        & $\boldsymbol{s}$ & $s_f$ & $s_f$ & $s_f$ & $s_f$ & $s_f$ & $s_f$ & $s_f$ & $s_f$ & $s_f$ & $s_f$ \\
        \TABLEmidruler
        \software{Panthera}     & \TABLEcellnumbold{3.7} & \TABLEcellnum{4.0} & \TABLEcellnum{4.0} & \TABLEcellnum{3.5} & \TABLEcellnum{4.0} & \TABLEcellnum{3.0} & \TABLEcellnum{4.5} & \TABLEcellnum{3.0} & \TABLEcellnum{3.0} & \TABLEcellnum{3.5} & \TABLEcellnum{4.5} \\
        \software{Project Cars} & \TABLEcellnumbold{3.7} & \TABLEcellnum{2.5} & \TABLEcellnum{2.5} & \TABLEcellnum{4.0} & \TABLEcellnum{2.5} & \TABLEcellnum{3.0} & \TABLEcellnum{4.5} & \TABLEcellnum{3.0} & \TABLEcellnum{5.0} & \TABLEcellnum{4.5} & \TABLEcellnum{4.5} \\
        \software{Unity}        & \TABLEcellnumbold{3.7} & \TABLEcellnum{3.0} & \TABLEcellnum{3.5} & \TABLEcellnum{4.0} & \TABLEcellnum{3.5} & \TABLEcellnum{3.0} & \TABLEcellnum{4.0} & \TABLEcellnum{3.5} & \TABLEcellnum{4.5} & \TABLEcellnum{4.0} & \TABLEcellnum{4.0} \\
        \software{OpenDS}       & \TABLEcellnumbold{3.3}& \TABLEcellnum{1.5} & \TABLEcellnum{4.0} & \TABLEcellnum{4.0} & \TABLEcellnum{4.5} & \TABLEcellnum{2.5} & \TABLEcellnum{3.0} & \TABLEcellnum{2.0} & \TABLEcellnum{5.0} & \TABLEcellnum{2.5} & \TABLEcellnum{4.5} \\
        \software{STISIM Drive} & \TABLEcellnumbold{2.5} & \TABLEcellnum{1.5} & \TABLEcellnum{3.0} & \TABLEcellnum{3.0} & \TABLEcellnum{4.5} & \TABLEcellnum{2.0} & \TABLEcellnum{3.0} & \TABLEcellnum{2.0} & \TABLEcellnum{2.0} & \TABLEcellnum{2.0} & \TABLEcellnum{2.5} \\
        \software{SCANeR}       & \TABLEcellnumbold{2.4} & \TABLEcellnum{1.5} & \TABLEcellnum{4.0} & \TABLEcellnum{4.0} & \TABLEcellnum{5.0} & \TABLEcellnum{2.0} & \TABLEcellnum{2.5} & \TABLEcellnum{2.0} & \TABLEcellnum{1.0} & \TABLEcellnum{2.0} & \TABLEcellnum{2.0} \\
        \software{VDrift}       & \TABLEcellnumbold{2.1} & \TABLEcellnum{1.5} & \TABLEcellnum{2.5} & \TABLEcellnum{3.0} & \TABLEcellnum{2.0} & \TABLEcellnum{1.5} & \TABLEcellnum{3.0} & \TABLEcellnum{2.0} & \TABLEcellnum{1.5} & \TABLEcellnum{2.0} & \TABLEcellnum{2.0} \\
        \TABLEbottomruler
    \end{tabularx}

	\caption{State of the Art: \glspl{framework} ranking (no bias)}\label{tb:stateoftheart:equal}
	\CAPTIONadd{This ranking is made without biasing neither \gls{practicability} or \gls{easiness}, and is thus a broader survey of each \gls{framework}'s overall capabilites.}
\end{table}

\begin{table}[!p]
    \begin{tabularx}{\FLOATtextwidth}{lY|*{10}{Y}}
        & \textbf{tot} & mod. & map & wor. & tra. & sen. & \FONTsmallcaps{ff} & tou. & \FONTsmallcaps{vr} & whe. & sea. \\
        \TABULARXpartialruler(lr){2-2} \TABULARXpartialruler(lr){3-3} \TABULARXpartialruler(lr){4-4} \TABULARXpartialruler(lr){5-5} \TABULARXpartialruler(lr){6-6} \TABULARXpartialruler(lr){7-7} \TABULARXpartialruler(lr){8-8} \TABULARXpartialruler(lr){9-9} \TABULARXpartialruler(lr){10-10} \TABULARXpartialruler(lr){11-11} \TABULARXpartialruler(lr){12-12}
        & $\boldsymbol{s}$ & $s_f$ & $s_f$ & $s_f$ & $s_f$ & $s_f$ & $s_f$ & $s_f$ & $s_f$ & $s_f$ & $s_f$ \\
        \TABLEmidruler
        \software{Unity}        & \TABLEcellnumbold{4.2} & \TABLEcellnum{3.8} & \TABLEcellnum{4.1} & \TABLEcellnum{4.4} & \TABLEcellnum{4.1} & \TABLEcellnum{3.8} & \TABLEcellnum{4.4} & \TABLEcellnum{4.1} & \TABLEcellnum{4.7} & \TABLEcellnum{4.4} & \TABLEcellnum{4.4} \\
        \software{Panthera}     & \TABLEcellnumbold{4.0} & \TABLEcellnum{4.4} & \TABLEcellnum{4.4} & \TABLEcellnum{3.7} & \TABLEcellnum{4.4} & \TABLEcellnum{3.4} & \TABLEcellnum{4.7} & \TABLEcellnum{3.4} & \TABLEcellnum{3.0} & \TABLEcellnum{3.7} & \TABLEcellnum{4.7} \\
        \software{Project Cars} & \TABLEcellnumbold{3.8} & \TABLEcellnum{2.7} & \TABLEcellnum{2.7} & \TABLEcellnum{4.0} & \TABLEcellnum{2.7} & \TABLEcellnum{3.4} & \TABLEcellnum{4.7} & \TABLEcellnum{3.4} & \TABLEcellnum{5.0} & \TABLEcellnum{4.7} & \TABLEcellnum{4.7} \\
        \software{OpenDS}       & \TABLEcellnumbold{3.4} & \TABLEcellnum{1.7} & \TABLEcellnum{4.0} & \TABLEcellnum{4.0} & \TABLEcellnum{4.7} & \TABLEcellnum{2.7} & \TABLEcellnum{3.0} & \TABLEcellnum{2.0} & \TABLEcellnum{5.0} & \TABLEcellnum{2.7} & \TABLEcellnum{4.7} \\
        \software{STISIM Drive} & \TABLEcellnumbold{2.5} & \TABLEcellnum{1.7} & \TABLEcellnum{3.0} & \TABLEcellnum{3.0} & \TABLEcellnum{4.7} & \TABLEcellnum{2.0} & \TABLEcellnum{3.0} & \TABLEcellnum{2.0} & \TABLEcellnum{2.0} & \TABLEcellnum{2.0} & \TABLEcellnum{2.7} \\
        \software{SCANeR}       & \TABLEcellnumbold{2.5} & \TABLEcellnum{1.7} & \TABLEcellnum{4.0} & \TABLEcellnum{4.0} & \TABLEcellnum{5.0} & \TABLEcellnum{2.0} & \TABLEcellnum{2.7} & \TABLEcellnum{2.0} & \TABLEcellnum{1.0} & \TABLEcellnum{2.0} & \TABLEcellnum{2.0} \\
        \software{VDrift}       & \TABLEcellnumbold{2.2} & \TABLEcellnum{1.7} & \TABLEcellnum{2.7} & \TABLEcellnum{3.0} & \TABLEcellnum{2.0} & \TABLEcellnum{1.7} & \TABLEcellnum{3.0} & \TABLEcellnum{2.0} & \TABLEcellnum{1.7} & \TABLEcellnum{2.0} & \TABLEcellnum{2.0} \\
        \TABLEbottomruler
    \end{tabularx}

    \caption{State of the Art: \glspl{framework} ranking (practicability bias)}\label{tb:stateoftheart:practicability}
	\CAPTIONadd{This ranking is made with a \gls{practicability} bias (\num{0,7} vs.\ \num{0,3}), and specifically ranks which \glspl{framework} have the most potential to implement the \glspl{feature}.}
\end{table}

\begin{table}[!p]
    \begin{tabularx}{\FLOATtextwidth}{lY|*{10}{Y}}
        & \textbf{tot} & mod. & map & wor. & tra. & sen. & \FONTsmallcaps{ff} & tou. & \FONTsmallcaps{vr} & whe. & sea. \\
        \TABULARXpartialruler(lr){2-2} \TABULARXpartialruler(lr){3-3} \TABULARXpartialruler(lr){4-4} \TABULARXpartialruler(lr){5-5} \TABULARXpartialruler(lr){6-6} \TABULARXpartialruler(lr){7-7} \TABULARXpartialruler(lr){8-8} \TABULARXpartialruler(lr){9-9} \TABULARXpartialruler(lr){10-10} \TABULARXpartialruler(lr){11-11} \TABULARXpartialruler(lr){12-12}
        & $\boldsymbol{s}$ & $s_f$ & $s_f$ & $s_f$ & $s_f$ & $s_f$ & $s_f$ & $s_f$ & $s_f$ & $s_f$ & $s_f$ \\
        \TABLEmidruler
        \software{Panthera}     & \TABLEcellnumbold{3.5} & \TABLEcellnum{3.6} & \TABLEcellnum{3.6} & \TABLEcellnum{3.3} & \TABLEcellnum{3.6} & \TABLEcellnum{2.6} & \TABLEcellnum{4.3} & \TABLEcellnum{2.6} & \TABLEcellnum{3.0} & \TABLEcellnum{3.3} & \TABLEcellnum{4.3} \\
        \software{Project Cars} & \TABLEcellnumbold{3.5} & \TABLEcellnum{2.3} & \TABLEcellnum{2.3} & \TABLEcellnum{4.0} & \TABLEcellnum{2.3} & \TABLEcellnum{2.6} & \TABLEcellnum{4.3} & \TABLEcellnum{2.6} & \TABLEcellnum{5.0} & \TABLEcellnum{4.3} & \TABLEcellnum{4.3} \\
        \software{Unity}        & \TABLEcellnumbold{3.2} & \TABLEcellnum{2.2} & \TABLEcellnum{2.9} & \TABLEcellnum{3.6} & \TABLEcellnum{2.9} & \TABLEcellnum{2.2} & \TABLEcellnum{3.6} & \TABLEcellnum{2.9} & \TABLEcellnum{4.3} & \TABLEcellnum{3.6} & \TABLEcellnum{3.6} \\
        \software{OpenDS}       & \TABLEcellnumbold{3.2} & \TABLEcellnum{1.3} & \TABLEcellnum{4.0} & \TABLEcellnum{4.0} & \TABLEcellnum{4.3} & \TABLEcellnum{2.3} & \TABLEcellnum{3.0} & \TABLEcellnum{2.0} & \TABLEcellnum{5.0} & \TABLEcellnum{2.3} & \TABLEcellnum{4.3} \\
        \software{STISIM Drive} & \TABLEcellnumbold{2.4} & \TABLEcellnum{1.3} & \TABLEcellnum{3.0} & \TABLEcellnum{3.0} & \TABLEcellnum{4.3} & \TABLEcellnum{2.0} & \TABLEcellnum{3.0} & \TABLEcellnum{2.0} & \TABLEcellnum{2.0} & \TABLEcellnum{2.0} & \TABLEcellnum{2.3} \\
        \software{SCANeR}       & \TABLEcellnumbold{2.4} & \TABLEcellnum{1.3} & \TABLEcellnum{4.0} & \TABLEcellnum{4.0} & \TABLEcellnum{5.0} & \TABLEcellnum{2.0} & \TABLEcellnum{2.3} & \TABLEcellnum{2.0} & \TABLEcellnum{1.0} & \TABLEcellnum{2.0} & \TABLEcellnum{2.0} \\
        \software{VDrift}       & \TABLEcellnumbold{2.0} & \TABLEcellnum{1.3} & \TABLEcellnum{2.3} & \TABLEcellnum{3.0} & \TABLEcellnum{2.0} & \TABLEcellnum{1.3} & \TABLEcellnum{3.0} & \TABLEcellnum{2.0} & \TABLEcellnum{1.3} & \TABLEcellnum{2.0} & \TABLEcellnum{2.0} \\
        \TABLEbottomruler
    \end{tabularx}

    \caption{State of the Art: \glspl{framework} ranking (easiness bias)}\label{tb:stateoftheart:easiness}
	\CAPTIONadd{This ranking is made with an \gls{easiness} bias (\num{0,7} vs.\ \num{0,3}), and specifically ranks which \glspl{framework} have the most expansive \glspl{api} to work with.}
\end{table}


As expected, \software{Unity} scores the highest (the only \gls{framework} to score above \num{4,0} across all analyses) when the \gls{feasibility} score is calculated with a \gls{practicability} bias: this reflects its highly adaptive and extensible nature, it being a generic game engine and not a pre-existing driving simulation engine. At the same time though, \software{Unity} scores poorly (\num{3,2}) when importance is given to easiness, due to the lack of high-level \glspl{api} and the need to build the simulator from the ground up.

At \num{4,0} and \num{3,8} respectively \software{Panthera} and \software{Project Cars} are close seconds in the \gls{practicability} bias classification, also scoring the highest (\num{3,5} both) when \gls{easiness} is concerned: the availability of high-level \glspl{api}, along with the pre-existing implementation of a number of highly weighted \glspl{feature} make for an ideal combination.

\software{Panthera}, \software{Project Cars}, and \software{Unity} are the highest scorers \num{3,7} when no bias is given in the \gls{feasibility} calculation, further proving their edge over the other options.

\software{STISIM Drive} scores low (\num{2,5} or below) on all classifications, due to its lack of focus on aspects relevant to the simulator; \software{SCANeR} scores identically, although it does so due to code access concerns. \software{VDrift} on the other hand suffers its open-source nature (\num{2,2} or below), with slow development times and the lack of a supporting company further undermining its possibilities.

Both \software{Project Cars} and \software{Panthera} appear to be the ideal solutions: the choice of the former though depends on the access level to the engine, as determined by the developers \gls{sms}; the latter is a more mature and supported professional option, although with less potential capabilities. \software{Unity} becomes the third choice only if enough time and work-hours are available to develop the simulator from the outright basis: everything from the physics engine to the peripherals support will have to be coded in their entirety. \software{OpenDS} is thus the realistic third choice, in case \software{Unity} is deemed a too time-intensive solution: it scores right behind \software{Project Cars} and \software{Unity} in the non-biased calculations, and is on par with \software{Unity} w.r.t.\ \gls{easiness}. It provides high-level \glspl{api}, existing scenario building options, and basic peripherals integration; the immersion potentiality is lower though, due to the basic graphics and audio rendering.

\section{Recap}\label{sc:stateoftheart:recap}

In \fref{sc:stateoftheart:features} ten different \glspl{feature} were presented, listing all those requirements the simulator must adhere to. In \fref{sc:stateoftheart:frameworks} five \glspl{framework} were analysed. In \fref{sc:stateoftheart:comparison} the value of each \gls{framework} were quantified through weighting and means, to generate three classifications: from those, a general analysis was performed and \software{Project Cars} or \software{Panthera} were highlighted as the best overall \gls{framework} for the development of the simulator, with \software{Unity} and \software{OpenDS} as the fall-back solutions.
