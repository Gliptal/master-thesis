\chapter{Engine Integration}\label{ch:integration}

\begin{keywords}
	me, middleware, packet, payload
\end{keywords}

As explained in \fref{sc:software:madnessengine}, the \gls{me} can be roughly divided into four tiers (or layers). A simulation tick or update can be summarized as a single loop, with all the layers exchanging data from top to bottom, until the rendering input is ready to be given to the \gls{os}. In reality, all the layers may run at different rates (fixed or variable), and each can gracefully handle unsynchronized data.

\section{Approach}\label{sc:integration:approach}

In order to visualize the vehicle traffic simulated by the \gls{ts}, rendering it within \gls{me}'s world, custom data must be provided at one of the layers, either replacing or supplementing the already existing data. Three possible approaches are presented.

\begin{itemize}
	\item \FONTbold{\acrshort{ai} splice} --- Supplying the \gls{ai} layer with custom \gls{ai} commands, basically telling the vehicles to exactly replicate their behaviour in the \gls{ts} through a set of goals and actions. Given the nature of the data received from the \gls{ts} (raw state vectors), this approach is clearly the hardest to achieve proper synchronization with: it would require abstracting the raw data to higher-level intent in real time, with little to no knowledge of what the \gls{ts} entitles goals are.
	\item \FONTbold{physics splice} --- Replacing the \gls{ai} supplied data before it is handled by the physics layer. This requires a thorough knowledge of the physics system, since the raw data obtained from the \gls{ts} should be converted in more abstract data on which the physics responses may be calculated on.
	\item \FONTbold{rendering splice} --- Replacing the data used by the rendering system is the easier, but most approachable solution: the raw data obtained from the \gls{ts} can be used directly (after trivial transformations), since the rendering system internally uses the same position vectors. The main drawback is that all the systems that receive data from the physics layer (e.g.\ sounds, pilot animations, car lights, advanced wheel physics) still receive their data from the original unmodified physics layer, resulting in unsynchronized states.
\end{itemize}

Due to time, support, and knowledge constraints the easier but more limited approach was chosen.

\begin{image}
	{integration/splice}{0.5}
	{rendering splice}
	{im:integration:splice}
	{}
\end{image}

Sending player data to the \gls{ts} is instead straightforward: telemetry metrics about the player (update on each tick) are available at all times within \gls{me}, and basic state vectors can be easily obtained and packaged for the \gls{middleware}.

\section{Traffic Module}\label{sc:integration:trafficmodule}

The integrating code is entirely contained in a separate \gls{me} module, with only the necessary modifications made directly on the rendering module. The methods exposed by this traffic module are then called appropriately to obtain the data required for correct rendering.

\begin{image}
	{integration/flow}{0.6}
	{traffic module flow}
	{im:integration:flow}
	{}
\end{image}

\subsubsection{Aids}

To aid in the process, a custom \code{Vehicle} extends the \gls{middleware}'s \code{VehicleEntity} class. This allows the encapsulation of all those methods, data, and matrices that are required to convert from the \gls{middleware}'s to \gls{me}'s format and back.\footnote{All of the following code snippets are removed of their comments and documentation.}

\begin{codelist}{Vehicle.hpp}{51}
class Vehicle : public UDPConnector::VehicleEntity {
public:
	enum ROLE_e {
		ROLE_PLAYER = 0,
		ROLE_TRAFFIC
	};

	struct Vectors_t {
		nsMaths::BVec3f location;
		nsMaths::BVec3f orientation;
		nsMaths::BVec3f velocity;
		nsMaths::BVec3f acceleration;

		nsMaths::BVec4f wheelPosition[MAX_WHEELS];
		nsMaths::BQuatf wheelRotation[MAX_WHEELS];
	} vectors;

	Vehicle();
	explicit Vehicle(const UDPConnector::VehicleEntity);
	explicit Vehicle(const Vehicle, const Vehicle);

	bool convert_vehicle_to_entity(const MWL::PhysicsMetrics*, const transformMatrix);
	bool convert_entity_to_vehicle(const UDPConnector::VehicleEntity*, const transformMatrix);

	void set_entity_metadata(const uint8_t, const UDPConnector::VehicleEntity::VehicleType_e);
	void set_entity_lights(const MWL::PhysicsMetrics* metrics, const MWL::SMS::ParticipantMemento);

private:
	struct Geometry_t {
		float axle;
		float wheelbase;
		float wheelRadius;
	};

	struct Geometries_t {
		Geometry_t sedan = {1.630f, 2.900f, 0.340f};
	} geometries;

	void set_wheel_position(const int);
	void set_wheel_rotation(const int);

	void set_entity_vector(UDPConnector::maths::Vector_t*, const nsMaths::BVec3f*, const transformMatrix);
	void set_vehicle_vector(nsMaths::BVec3f*, const UDPConnector::maths::Vector_t*, const transformMatrix);
};
\end{codelist}

An history matrix is used to hold the data about the traffic entities received from the \gls{ts}. Depending on the value of \code{IOTECH_TRAFFIC_DATA_HISTORY}, this matrix may either keep only the latest received \code{Vehicle}(s), or keep an history of past \code{Vehicle}(s) up to a specified number of past receivals.\footnote{Currently, only no history or one past receival are implemented}

\begin{codelist}{Traffic.hpp --- m\_trafficVehicles}{241}
	static Vehicle* m_trafficVehicles[IOTECH_TRAFFIC_DATA_HISTORY][IOTECH_TRAFFIC_MAX_VEHICLES];
\end{codelist}

See \fref{ax:code:historymatrix} for an explanation of how the history matrix works and an example of the documentation.

\subsubsection{Initialization}

Instantiating the traffic module generates a new \gls{middleware} \code{UDPConnector} object, that will handle all the communication with the external \gls{ts}. Once instantiated, the module is initialized by opening the required connections with the \gls{ts} (either receive only, or both send and receive). Note that this module (and by extension, \gls{me}) has no knowledge on what sockets or network protocol is used to obtain and send the data: only the \glspl{ip} of the machines hosting \gls{me} and the \gls{ts}, and the respective process ports are required.

\begin{codelist}{Traffic.cpp --- init()}{73}
void Traffic::init(const INIT_RESOURCE_e resource) {
    switch (resource) {
        case INIT_NONE:
            break;
        case INIT_RX:
            m_connector->open_RX_connection(UDPCONNECTOR_VTD_DEFAULT_RX_IP, UDPCONNECTOR_VTD_DEFAULT_RX_PORT);
            break;
        case INIT_BOTH:
        default:
            m_connector->open_RX_connection(UDPCONNECTOR_VTD_DEFAULT_RX_IP, UDPCONNECTOR_VTD_DEFAULT_RX_PORT);
            m_connector->open_TX_connection(UDPCONNECTOR_VTD_DEFAULT_TX_IP, UDPCONNECTOR_VTD_DEFAULT_TX_PORT);
            break;
    }
}
\end{codelist}

\subsubsection{Update}

\begin{image}
	{integration/update}{1.0}
	{traffic module update}
	{im:integration:update}
	{draft=false}
\end{image}

An \code{update()} function handles everything required to update the data about all the \gls{ts} entities. Each time it's called, the player data is first sent to the \gls{ts}, and then the traffic data from the \gls{ts} is handled:

\begin{itemize}
	\item data about the player car is taken from \gls{me}'s metrics (through the \code{ParticipantConstPtr} and \code{ParticipantMemento} objects), and converted into the format required by the \gls{middleware};
	\item an \code{std::vector} of \code{TrafficEntity} with the \code{VehicleEntity} representing the \gls{me} player is used to prepare the payload;
	\item the packet is sent to the \gls{ts}.
	\item a packet is received from the \gls{ts};
	\item the payload is converted into appropriate \code{TrafficEntity}, and of those the data about the player is filtered (since the \gls{ts} may send it back any other generic entity);
	\item for every \code{TrafficEntity}, its data is converted into the format required by \gls{me};
	\item the resulting \code{Vehicle} is appropriately placed into the history matrix.
\end{itemize}

\begin{codelist}{Traffic.cpp --- update()}{88}
void Traffic::update(const MWL::SMS::ParticipantConstPtr participant, const MWL::SMS::ParticipantMemento memento) {
    PhysicsMetrics* metrics = participant->GI()->GetRenderTimePhysicsMetrics();
    if (!metrics) {
        return;
    }

    UDPConnector::VehicleEntity* playerEntity = new Vehicle();

    ((Vehicle*) playerEntity)->convert_vehicle_to_entity(metrics, transformMatrices.METoSAE);
    ((Vehicle*) playerEntity)->set_entity_metadata(Vehicle::ROLE_PLAYER, Vehicle::TYPE_SEDAN);
    ((Vehicle*) playerEntity)->set_entity_lights(metrics, memento);

    std::vector<UDPConnector::TrafficEntity*> entitiesTX;
    entitiesTX.push_back(playerEntity);

    if (m_connector->prepare_TX_payload(entitiesTX)) {
        m_connector->send_packet();
    }

    entitiesTX.clear();
    entitiesTX.shrink_to_fit();
    delete playerEntity;

    if (m_connector->receive_packet()) {
#if IOTECH_TRAFFIC_DATA_HISTORY >= 2
        m_isNewTrafficVehicles = true;
#endif

        std::vector<UDPConnector::TrafficEntity*> allEntities     = m_connector->handle_RX_payload();
        std::vector<UDPConnector::VehicleEntity*> vehicleEntities = filter_AI_vehicles(allEntities);

        for (std::vector<UDPConnector::VehicleEntity*>::iterator iter = vehicleEntities.begin(); iter != vehicleEntities.end(); iter++) {
            Vehicle* vehicle = new Vehicle(**iter);

            vehicle->convert_entity_to_vehicle(*iter, transformMatrices.SAEToME);

#if IOTECH_TRAFFIC_DATA_HISTORY >= 2
            if (!is_no_history(vehicle->id)) {
                delete m_trafficVehicles[1][vehicle->id];
            }
            m_trafficVehicles[1][vehicle->id] = m_trafficVehicles[0][vehicle->id];
#endif
            m_trafficVehicles[0][vehicle->id] = vehicle;
        }
    }
}
\end{codelist}

\subsubsection{Getting the Vehicle}

Once the network has been polled, the history matrix may have been updated with new \code{Vehicle}(s), or the older \code{Vehicle}(s) may be still there (depending on the differences in \gls{me}'s tickrate and the \gls{ts}'s sendrate). Obtaining the latest possible vehicle changes depending on the value of \code{IOTECH_TRAFFIC_DATA_HISTORY}. With \num{1} the history matrix keeps only one instance for each entity at all times, so that instance is returned; with \num{2} (or possibly more) a variable helps in taking into account whether the most recent \code{Vehicle} comes from the \gls{ts} or from extrapolation. If the former, then \code{m_isNewTrafficVehicles} is set to \code{false}, to signal that subsequent calls without a successful network poll require extrapolation; if the latter, then a "ghost" \code{Vehicle} is extrapolated and the history matrix is updated accordingly by bumping down the other \code{Vehicle}(s).

\begin{codelist}{Traffic.hpp --- get\_latest\_vehicle()}{135}
Vehicle* Traffic::get_latest_vehicle(const int vehicleId) {
#if IOTECH_TRAFFIC_DATA_HISTORY >= 2
    if (m_isNewTrafficVehicles || is_no_history(vehicleId)) {
        m_isNewTrafficVehicles = false;
    } else {
        Vehicle* ioVehicle = new Vehicle(*(m_trafficVehicles[0][vehicleId]), *(m_trafficVehicles[1][vehicleId]));

        delete m_trafficVehicles[1][vehicleId];
        m_trafficVehicles[1][vehicleId] = m_trafficVehicles[0][vehicleId];
        m_trafficVehicles[0][vehicleId] = ioVehicle;
    }
#endif

    return m_trafficVehicles[0][vehicleId];
}
\end{codelist}

\subsubsection{Extrapolation}

When \code{IOTECH_TRAFFIC_DATA_HISTORY} is set to \num{2} or more, extrapolation is made available for rendering purposes. If the \gls{ts} sendrate is less than \gls{me}'s tickrate, an entity may be rendered in the same state multiple times, since a packet with the updated data hasn't been received yet. In this case, a "ghost" \code{Vehicle} is generated by extrapolating from the \code{Vehicle}(s) in the history matrix.

The calculations themselves are made by taking the current and previous \code{Vehicle}, and instantiating a third \code{Vechicle} by extrapolating possible future position vectors. Only location vectors are extrapolated: orientation, velocity, and acceleration are set at the current values. See \fref{ax:math:extrapolation} for the mathematics behind the extrapolation algorithm.

\begin{codelist}{Vehicle.cpp --- Vehicle()}{56}
Vehicle::Vehicle(const Vehicle current, const Vehicle old) {
    id         = current.id;
    type       = current.type;
    components = current.components;

    vectors.location = current.vectors.location + ((current.vectors.location - old.vectors.location) * 0.3f);

    vectors.orientation  = current.vectors.orientation;
    vectors.velocity     = current.vectors.velocity;
    vectors.acceleration = current.vectors.acceleration;

    for (int wheel = 0; wheel < MAX_WHEELS; wheel++) {
        vectors.wheelPosition[wheel] = current.vectors.wheelPosition[wheel];
        vectors.wheelRotation[wheel] = current.vectors.wheelRotation[wheel];
    }
}
\end{codelist}

\section{Limitations}\label{sc:integration:limitations}

\section{Recap}\label{sc:integration:recap}
