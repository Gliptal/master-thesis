\chapter{Middleware Development}\label{ch:middleware}

\begin{keywords}
	middleware, netcode, packet, payload, proxy
\end{keywords}

The \gls{middleware} is a \planguage{c++} static library (\code{.lib}), developed with \software{VSCode} and \software{Microsoft Visual Studio}.

\section{Features}\label{sc:middleware:features}

The list of requirements was proposed and expanded upon by \company{ioTech}, based on the commercial viability of the final product (of which this \gls{middleware} is only a portion). The overall goal is to provide a public set of \glspl{api}, such that integration of any \gls{ts} could be done without having access to \gls{me}'s source code. Any \nth{3}--party company interested in using a \gls{ts} of their choice, could this seamlessly integrate it with minimal coding efforts and limited support from \company{ioTech}.

The current iteration of the \gls{middleware} achieves all of the following \glspl{feature}.

\begin{itemize}
	\item \FONTbold{black box} --- The \gls{middleware} should, by all intents and purposes, be a black box to any \nth{3}--party; only those parts required to successfully integrate a \gls{ts} should be grey boxed.
	\item \FONTbold{trivial} --- Implementing any \gls{ts} should be a simple, trivial, and documented process. Allowing a \nth{3}--party to independently develop the integration greatly reduces the support \software{ioTech} is expected to provide.
	\item \FONTbold{agnosticism} --- The \gls{netcode} part of the \gls{middleware} should be entirely hidden behind high-level functions: everything from sockets to memory handling should be handled inside the black box.
	\item \FONTbold{standardization} --- Vehicular standards, unit of measures, coordinate systems, and more should not refer to \gls{me}'s specific implementation, but rather to an industry accepted standard.\footnote{See \fref{sc:software:middleware}.}
\end{itemize}

The following \glspl{feature} are not fully supported yet, although the code is in place to allow an efficient transition to full support.

\begin{itemize}
	\item \FONTbold{plug-and-play} --- Switching between different \gls{ts} should be allowed by means of \gls{me} configuration files or in-game settings.
\end{itemize}

\section{Structure}\label{sc:middleware:structure}

The project is made of one core class, two helper namespaces, three data classes, and one required integration class.

\subsubsection{Helper}

\begin{filelist}
	logger.cpp
\end{filelist}

Provides utility functions to log the \gls{middleware}'s state and debug data. Logging is done on the \code{stderr} stream.

\begin{filelist}
	maths.cpp
\end{filelist}

Provides some basic maths functionality, mainly focused on matrix transformations between and within 3D systems. Among basic conversion functions, it defines a \code{Vector_t} struct, and an \code{apply_matrix} function: this is used to convert position (velocity, acceleration) vectors between coordinate systems. See \fref{ax:math:matrix} for the relevant maths basics.

\begin{codelist}{1}
	Vector_t apply_matrix(const Vector_t* vector, const transformMatrix matrix) {
		Vector_t transformed;

		transformed.x = vector->x*matrix[0][0] + vector->y*matrix[0][1] +
						vector->z*matrix[0][2] + 1*matrix[0][3];
		transformed.y = vector->x*matrix[1][0] + vector->y*matrix[1][1] +
						vector->z*matrix[1][2] + 1*matrix[1][3];
		transformed.z = vector->x*matrix[2][0] + vector->y*matrix[2][1] +
						vector->z*matrix[2][2] + 1*matrix[2][3];

		return transformed;
	}
\end{codelist}

The functions in this namespace are exposed as \glspl{api} to \nth{3}--parties through the \code{maths.hpp} header.

\subsubsection{Data}

The three data classes represent generic traffic objects, and must be used in the integration process.

\begin{filelist}
	LightEntity.cpp
	TrafficEntity.cpp
	VehicleEntity.cpp
\end{filelist}

The data obtained from the \gls{ts} must be serialized in the appropriate class (e.g.\ data on an \gls{ai} vehicle is serialized in a \code{VehicleEntity} instance); the serialized instances obtained from \gls{me} must be deserialized into the appropriate \gls{ts} format (e.g.\ a \code{LightEntity} is deserialized into traffic lights raw data). This serialization/deserialization is in fact the only code that a \nth{3}--party must produce for the integration.

\code{TrafficEntity} is the parent class, that contains values common to all other entities (like position and id); \code{LightEntity} and \code{VehicleEntity} both extend the base class, each adding the data relevant to their function (e.g.\ \code{VehicleEntity} adds velocity and acceleration vectors, among other values).

The headers are fully documented to ease the coding by \nth{3}--parties.

\subsubsection{Core}

The core class is a pure virtual class, which means it must be extended by a child class that implements its pure virtual methods. \code{UDPConnector} silently handles the netcode, and only leaves the conversion to/from the \gls{ts} and \gls{me} to be implemented. As such the two pure virtual methods are \code{handle_RX_payload()} and \code{prepare_TX_payload()}, that respectively must handle the \gls{payload} received from the \gls{ts}, and prepare the data to be sent to the \gls{ts}.

Summarily, the execution flow is as follows:

\begin{itemize}
	\item an instance of \code{UDPConnector} is created by the implementing child class, via the \code{UDPConnector()} constructor;
	\item receiving and transmitting connections are opened between the \gls{ts} and \gls{me}, via \code{open_RX_connection()} and \code{open_TX_connection()};
	\item a \gls{packet} is received from the \gls{ts}, via \code{receive_packet()}, and placed in \code{m_payloadRX}
	\item the \gls{payload} is converted/serialized by \code{handle_RX_payload()}, and the returned \code{std::vector} consumed by \gls{me} for rendering;
	\item player data provided by \gls{me} is converted/deserialized by \code{handle_TX_payload()}, and the resulting \gls{payload} is placed in \code{m_payloadTX};
	\item a \gls{packet} is sent to the \gls{ts} with the \gls{payload} in \code{m_payloadTX}, via \code{send_packet()}.
\end{itemize}

\begin{image}
	{middleware/flow}{0.8}
	{\gls{middleware} execution flow}
	{im:middleware:flow}
	{}
\end{image}

A destructor is also used to properly close the connections and free all the used memory. Note that there is no requirement for a 1:1 relation between \glspl{packet} sent and \glspl{packet} received: both may happen independently and at different rates.

These methods are called within \gls{me}, as part of the integration with the game engine (\fref{ch:integration}).

\subsubsection{Integration}

Integrating any \gls{ts} into the \gls{middleware} (and thus \gls{me}) simply requires implementing \code{handle_RX_payload()} and \code{prepare_TX_payload()}. The former should take the \gls{payload} from the \gls{ts}, generate a list of matching \code{TrafficEntity}, and return them; how this is achieved is irrelevant from the \gls{middleware}'s perspective. The latter should take a list of \code{TrafficEntity} (more specifically, a \code{VehicleEntity} representing the player), covert it into the appropriate byte format required by the \gls{ts}, and place it into \code{m_payloadTX}.

\section{Netcode}\label{sc:middleware:netcode}

\begin{definition}{netcode}
\end{definition}

\begin{definition}{packet}
\end{definition}

\begin{definition}{payload}
\end{definition}

The netcode is fully handled by the core \code{UDPConnector} class: more specifically, by the methods detailed in \fref{tb:middleware:netcode}.

\begin{table}[!ht]
	\centering
    \begin{tabular}{llll}
    	\TABLEmulticolumn{1}{c}{method} & \TABLEmulticolumn{1}{c}{returns} &\TABLEmulticolumn{1}{c}{parameters} & \TABLEmulticolumn{1}{c}{effect} \\
		\TABLEmidruler
        \code{open_RX_connection()} & success/failure & \code{ip:port} & opens a receive socket on the \code{ip:port} \\
        \code{open_TX_connection()} & success/failure & \code{ip:port} & opens a send socket on the \code{ip:port} \\
        \code{receive_packet()} & success/failure  & & receives a \gls{packet} and saves it in \code{m_payloadRX} \\
        \code{send_packet()} & success/failure  & & sends the \gls{packet} saved in \code{m_payloadTX}  \\
        \TABLEbottomruler
    \end{tabular}

    \caption{\gls{netcode} methods}\label{tb:middleware:netcode}
\end{table}


Note that all the following code snippets have comments, debugging, and cross-compiler options removed for functional clarity. See \fref{ax:code:fullcode} for an example of unmodified code.

\subsection{\FONTsmallcaps{os} Sockets}

Windows \gls{os} sockets are provided by the\code{<Ws2tcpip.h>} standard \planguage{c} header. Socket references are saved in \code{int} variables, that represent a file descriptor for that specific socket.

\subsubsection{Receival Connection}

An in (receiving) socket is represented by a \code{strucy sockaddr_in}. The file descriptor is obtained with a \code{socket()} call, with the \code{IPPROTO_UDP} option specifying the \gls{udp} protocol; options are specified with \code{setsockopt()}. Properties of the socket (like \gls{ip} and port) are directly set to the \code{struct}'s appropriate fields. The sockets is set as non-blocking (i.e.\ the execution continues if nothing is found on a \code{recv} call) via \code{ioctlsocket()} and the \code{FIONBIO} option. Finally, the socket is opened (bound) with \code{bind}, by passing the file descriptor and the socket \code{struct}.

\begin{codelist}{1}
bool UDPConnector::open_RX_connection(const char* ip, const uint16_t port) {
    if (m_socketRXfd > -1)
        return false;

    struct sockaddr_in sock;
    uint32_t address;

    int opt;
    int result;

	InetPton(AF_INET, ip, &address);

    m_socketRXfd = socket(AF_INET, SOCK_DGRAM, IPPROTO_UDP);

    if (m_socketRXfd == -1)
        return false;

    opt = 1;
    result = setsockopt(m_socketRXfd, SOL_SOCKET, SO_REUSEADDR, (char*) &opt, sizeof(opt));

    if (result == -1)
        return false;

    memset(&sock, 0, sizeof(sock));
    sock.sin_family      = AF_INET;
    sock.sin_addr.s_addr = address;
    sock.sin_port        = htons(port);

    result = bind(m_socketRXfd, (struct sockaddr*) &sock, sizeof(struct sockaddr_in));

    if (result == -1)
        return false;

    opt = 1;
	ioctlsocket(m_socketRXfd, FIONBIO, (u_long*) &opt);

    return true;
}
\end{codelist}

\subsubsection{Delivery Connection}

Opening an out (delivery) socket is similar, with only minor changes in the options for the socket. The major difference is the usage of \code{connect} instead of \code{bind}. The former is used to connect to a remote \code{ip:port}, whilst the latter provides an endpoint for external connections.

\begin{codelist}{1}
bool UDPConnector::open_TX_connection(const char* ip, const uint16_t port) {
    if (m_socketTXfd > -1)
        return false;

    struct sockaddr_in sock;
    uint32_t address;

    int opt;
    int result;

    InetPton(AF_INET, ip, &address);

    m_socketTXfd = socket(AF_INET, SOCK_DGRAM, 0);

    if (m_socketTXfd == -1)
        return false;

    opt = 1;
    result = setsockopt(m_socketTXfd, SOL_SOCKET, SO_REUSEADDR, (char*) &opt, sizeof(opt));

    if (result == -1)
        return false;

    opt = 1;
    result = setsockopt(m_socketTXfd, SOL_SOCKET, SO_BROADCAST, (char*) &opt, sizeof(opt));

    if (result == -1)
        return false;

    memset(&sock, 0, sizeof(sock));
    sock.sin_family      = AF_INET;
    sock.sin_addr.s_addr = address;
    sock.sin_port        = htons(port);

    result = connect(m_socketTXfd, (struct sockaddr*) &sock, sizeof(struct sockaddr_in));

    if (result == -1)
        return false;

    return true;
}
\end{codelist}

\subsubsection{Receiving Packet}

The \code{recv()} function polls the socket to see if any \glspl{packet} were sent to it. If one is found the \gls{packet}'s \gls{payload} is saved in \code{m_payloadRX}.

\begin{codelist}{1}
bool UDPConnector::receive_packet() {
    int payloadSize = recv(m_socketRXfd, m_payloadRX, UDPCONNECTOR_ME_PAYLOAD_BUFFER, 0);

	if (payloadSize > 0)
	    return true;
    else
        return false;
}
\end{codelist}

\subsubsection{Sending Packet}

The \code{send()} function takes a \gls{payload} (and its size) and sends a \gls{packet} containing it towards the specified remote \code{ip:port}.

\begin{codelist}{1}
bool UDPConnector::send_packet() {
    int result = send(m_socketTXfd, m_payloadTX, m_payloadTXSize, 0);

    if (result > 0)
        return true;
    else
        return false;
}
\end{codelist}

\subsection{0mq Sockets}

\software{0mq} sockets are provided by the \code{<zmq.h>} library \planguage{c} header. Socket references are saved as \code{void} pointers, that represent the memory allocation of that specific socket. A context must be instantiated as well before creating the sockets, that contains some general configuration settings about the \software{0mq} middleware process.

The \code{UDPConnector()} constructor instantiates a new context, and specifies the receive and send socket references. The current implementation uses the \FONTsmallcaps{pub/sub} model: a receiving socket subscribes to a remote publishing socket; multiple receiving sockets may subscribe to a single publish socket at the same time. The project's structure would only require a single subscriber, but the 1:1 \software{0mq} \FONTsmallcaps{req/rep} forces the server and client sendrates to be the same (each request must be followed by a reply).

\begin{codelist}{63}
UDPConnector::UDPConnector() {
    m_context  = zmq_ctx_new();
    m_socketRX = zmq_socket(m_context, ZMQ_SUB);
    m_socketTX = zmq_socket(m_context, ZMQ_PUB);

    m_payloadRX = (char*) malloc(UDPCONNECTOR_ME_PAYLOAD_BUFFER);
    m_payloadTX = (char*) malloc(UDPCONNECTOR_ME_PAYLOAD_BUFFER);
}
\end{codelist}

\subsubsection{Receival Connection}

Opening a receival connection in \software{0mq} requires only a \code{connect} call, specifying the socket reference and the remote \code{ip:port} to connect to. \code{zmq_setsockopt} allows setting some options:

\begin{itemize}
	\item \code{ZMQ_CONFLATE} forces \software{0mq} to only retain the latest \gls{packet} received, and not keep a queue of received but not handled old \glspl{packet};
	\item \code{ZMQ_SUBSCRIBE} specifies the list the socket subscribes to (in this case, only one unnamed list exists).
\end{itemize}

\begin{codelist}{1}
bool UDPConnector::open_RX_connection(const char* ip, const uint16_t port) {

    std::string addr = std::string(ZMQBRIDGE_REMOTE_PROTOCOL) + "://" + std::string(ip) + ":" + std::to_string(port);

    int conflate = 1;
    zmq_setsockopt(m_socketRX, ZMQ_CONFLATE, &conflate, sizeof(conflate));
    zmq_connect(m_socketRX, addr.c_str());
    zmq_setsockopt(m_socketRX, ZMQ_SUBSCRIBE, "", 0);

    return true;
}
\end{codelist}

\subsubsection{Delivery Connection}

A delivery connection is set up with the \code{bind} call, that creates a socket to which remote \software{0mq} processes can connect to. The \code{ZMQ_CONFLATE} option is also passed to avoid a \gls{packet} queueing.

\begin{codelist}{1}
bool UDPConnector::open_TX_connection(const char* ip, const uint16_t port) {
    std::string addr = std::string(ZMQBRIDGE_REMOTE_PROTOCOL) + "://" + std::string(ip) + ":" + std::to_string(port);

    int conflate = 1;
    zmq_setsockopt(m_socketTX, ZMQ_CONFLATE, &conflate, sizeof(conflate));
    zmq_bind(m_socketTX, addr.c_str());

    return true;
}
\end{codelist}

\subsubsection{Receiving Packet}

Receiving a \gls{packet} on a socket requires a single \code{zmq_recv()} call, specifying the \code{ZMQ_NOBLOCK} option for a non-blocking receive (the execution does not halt until a \gls{packet} is received). The \gls{payload} is placed in \code{m_payloadRX}.

\begin{codelist}{1}
bool UDPConnector::receive_packet() {
    zmq_recv(m_socketRX, m_payloadRX, UDPCONNECTOR_ME_PAYLOAD_BUFFER, ZMQ_NOBLOCK);

    if (std::string(m_payloadRX).size() > 0)
        return true;
    else
        return false;
}
\end{codelist}

\subsubsection{Sending Packet}

Sending a \gls{packet} requires firstly sending the list name for that \gls{packet} (in this case, an empty name), and then adding the \gls{payload}. Both are done with the \code{zmq_send()} function, with the first part specifying \code{ZMQ_SENDMORE} to notify the receiver that more bytes from the same \gls{packet} are to be expected. The \gls{payload} sent is the one found in \code{m_payloadTX}.

\begin{codelist}{1}
bool UDPConnector::send_packet() {
	zmq_send(m_socketTX, "", 0, ZMQ_SNDMORE);
	int sentBytes = zmq_send(m_socketTX, m_payloadTX, m_payloadTXSize, 0);

	if (sentBytes > 0)
		return true;
	else
		return false;
}
\end{codelist}

\section{Testing}\label{sc:middleware:testing}

\section{Recap}\label{sc:middleware:recap}
