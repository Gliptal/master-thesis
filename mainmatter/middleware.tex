\chapter{Middleware Development}\label{ch:middleware}

\begin{keywords}
	middleware, netcode, packet, payload, proxy
\end{keywords}

The \gls{middleware} is a \planguage{c++} static library (\file{.lib}), developed with \software{VSCode} and \software{Microsoft Visual Studio}. This library is added statically at \gls{me}'s compile time.

\section{Features}\label{sc:middleware:features}

The list of requirements was initially requested by\company{ioTech}, and further expanded during the development process. The main goal is achieving commercial viability of the final product (of which this \gls{middleware} is only a portion), by provide a public set of \glspl{api} and code bindings, such that integrating any \gls{ts} could be done without having access to \gls{me}'s (or whichever engine's) source code. Any \nth{3}--party company interested in using the \gls{ts} of their choice, could thus seamlessly integrate it with reduced coding effort and limited support from \company{ioTech}.

The current iteration of the \gls{middleware} achieves the following \glspl{feature}:

\begin{itemize}
	\item \FONTbold{black box} --- The \gls{middleware} presents itself as a black box to any \nth{3}--party; only those code bindings required to successfully integrate a \gls{ts} are grey boxed.
	\item \FONTbold{trivial} --- Implementing any \gls{ts} is a (relatively) simple, trivial, and documented process. Allowing a \nth{3}--party to independently develop the integration greatly reduces the support \software{ioTech} is expected to provide.
	\item \FONTbold{agnosticism} --- The \gls{netcode} part of the \gls{middleware} is be entirely hidden behind high-level functions: everything from sockets to memory handling is handled inside the black box, hidden to the \nth{3}-party.
	\item \FONTbold{standardization} --- Vehicular standards, unit of measures, coordinate systems, and more are not \gls{me}'s specifications, but rather an industry accepted standard.\footnote{See \fref{sc:software:middleware}.}
\end{itemize}

\FLOATnoindent The following \glspl{feature} are not fully supported yet, although the code is in place to allow an efficient transition to full support.

\begin{itemize}
	\item \FONTbold{plug-and-play} --- Switching between different \gls{ts} should be allowed by means of configuration files or in-game settings.
\end{itemize}

\section{Structure}\label{sc:middleware:structure}

The \gls{middleware} is made of one core class, two helper namespaces, three data classes (one generic and two extensions), and one integration class.

\subsubsection{Helper}

\begin{filelist}
	\item logger.cpp
\end{filelist}

Contains a \code{namespace} that provides utility functions to log the \gls{middleware}'s state and debug the network payloads. Logging is entirely done on the \code{stderr} stream, and may be redirected to other sources by redirecting \code{stderr} itself.

\begin{filelist}
	\item maths.cpp
\end{filelist}

Contains a \code{namespace} that provides some basic maths functionality, mainly focused on matrix transformations between and within \FONTsmallcaps{3d} systems. Together with basic conversion functions, it defines a \code{Vector_t} struct, and an \code{apply_matrix()} function: the latter is used to convert state vectors (position, velocity, acceleration) between different coordinate systems. See \fref{ax:math:matrix} for some maths basics on the subject.

\begin{codelist}{maths.cpp --- apply\_matrix()}{46}
Vector_t apply_matrix(const Vector_t* vector, const transformMatrix matrix) {
	Vector_t transformed;

	transformed.x = vector->x*matrix[0][0] + vector->y*matrix[0][1] + vector->z*matrix[0][2] + 1*matrix[0][3];
	transformed.y = vector->x*matrix[1][0] + vector->y*matrix[1][1] + vector->z*matrix[1][2] + 1*matrix[1][3];
	transformed.z = vector->x*matrix[2][0] + vector->y*matrix[2][1] + vector->z*matrix[2][2] + 1*matrix[2][3];

	return transformed;
}
\end{codelist}

All the functions in \file{maths.cpp} are exposed as part of the \gls{api} to \nth{3}--parties through the \file{maths.hpp} header.

\subsubsection{Data}

Three data classes represent generic traffic objects, and are to be used when converting between the \gls{ts}'s and the \gls{middleware}'s standard.

\begin{filelist}
	\item LightEntity.cpp
	\item TrafficEntity.cpp
	\item VehicleEntity.cpp
\end{filelist}

The \glspl{payload} obtained from the \gls{ts} must be serialized in the appropriate class (e.g.\ data on an \gls{ai} vehicle is serialized in a \code{VehicleEntity} instance); similarly the game engine returns serialized instances that must be deserialized into the appropriate \gls{ts} format (e.g.\ a \code{LightEntity} is deserialized into raw traffic lights data). This serialization/deserialization process is the only code effort that a \nth{3}--party must produce to integrate any \gls{ts} in the game engine.

\code{TrafficEntity} is the generic parent class, that contains any values that are common to all other entities (like an identification number and position coordinates); \code{LightEntity} and \code{VehicleEntity} both extend this base class, each adding some data relevant to their specific function (e.g.\ \code{VehicleEntity} adds velocity and acceleration vectors).

The headers are fully documented to the  benefit of \nth{3}--parties.

\subsubsection{Core}

The core class is a \planguage{c++} pure virtual class, i.e.\ it cannot be instantiated, and must instead be extended by a child class that implements its pure virtual methods. \code{UDPConnector} silently handles the netcode portion of the \gls{middleware}, and only leaves the conversion between standards to be implemented. The two pure virtual methods that accomplish this conversion (or, (de)serialize the data) are \code{handle_RX_payload()} and \code{prepare_TX_payload()}: the former must handle the \gls{payload} received from the \gls{ts}, serializing it into proper data instances; the latter prepares the \gls{payload} to be sent to the \gls{ts}, with an opposite process.

Summarily, the \gls{middleware}'s execution flow is as follows:

\begin{itemize}
	\item an instance of \code{UDPConnector} is created through an implemented child class, using the \code{UDPConnector()} constructor;
	\item receival and transmission connections are opened between the \gls{ts} and the \gls{middleware}, via \code{open_RX_connection()} and \code{open_TX_connection()};
	\item a \gls{packet} is received from the \gls{ts}, via \code{receive_packet()}, and placed in \code{m_payloadRX}
	\item the \gls{payload} is converted/serialized by \code{handle_RX_payload()}, and the returned \code{std::vector} consumed by the game engine for rendering;
	\item player data provided by the game engine is converted/deserialized by \code{handle_TX_payload()}, and the resulting \gls{payload} is placed in \code{m_payloadTX};
	\item a \gls{packet} containing the \gls{payload} in \code{m_payloadTX} is sent to the \gls{ts}, via \code{send_packet()}.
\end{itemize}

\begin{image}
	{middleware/flow}{0.8}
	{Middleware Development: \gls{middleware} execution flow}
	{im:middleware:flow}
	{}
	{The internal execution flow of the \gls{middleware}, seen from the point of view of methods calls and fields access; the destructor is not shown. The incoming and outgoing flows are separate and independent with each other.}
\end{image}

A destructor also properly closes the connections and frees all the used memory. There is no requirement for a 1:1 relation between \glspl{packet} sent and \glspl{packet} received: the \gls{middleware} can gracefully handle independent and different rates.

These methods are part of the \gls{api} exposed to the game engine, as part of the integration process. See \fref{ch:integration} for the specific \gls{me} integration.

\subsubsection{Integration}

Integrating any \gls{ts} into the \gls{middleware} (and thus the game engine) only requires the implementation of the \code{handle_RX_payload()} and \code{prepare_TX_payload()} methods. The former should use the \glspl{payload} received from the \gls{ts}, generate a list of matching \code{TrafficEntity}, and return them. The latter should take a list of \code{TrafficEntity} (more specifically, a \code{VehicleEntity} representing the player), convert it into the appropriate format required by the \gls{ts}, and place it into \code{m_payloadTX}. How these conversions are achieved is irrelevant from the \gls{middleware}'s perspective. \fref{sc:middleware:implementation} is an example of such an integration.

\section{Netcode}\label{sc:middleware:netcode}

\begin{definition}{netcode}
\end{definition}

\begin{definition}{packet}
\end{definition}

\begin{definition}{payload}
\end{definition}

The netcode is handled entirely by the core \code{UDPConnector} class: more specifically, by the methods detailed in \fref{tb:middleware:netcode}.

\begin{table}[!ht]
	\centering
    \begin{tabular}{llll}
    	\TABLEmulticolumn{1}{c}{method} & \TABLEmulticolumn{1}{c}{returns} &\TABLEmulticolumn{1}{c}{parameters} & \TABLEmulticolumn{1}{c}{effect} \\
		\TABLEmidruler
        \code{open_RX_connection()} & success/failure & \code{ip:port} & opens a receive socket on the \code{ip:port} \\
        \code{open_TX_connection()} & success/failure & \code{ip:port} & opens a send socket on the \code{ip:port} \\
        \code{receive_packet()} & success/failure  & & receives a \gls{packet} and saves it in \code{m_payloadRX} \\
        \code{send_packet()} & success/failure  & & sends the \gls{packet} saved in \code{m_payloadTX}  \\
        \TABLEbottomruler
    \end{tabular}

    \caption{\gls{netcode} methods}\label{tb:middleware:netcode}
\end{table}


Note that all of the following code snippets have comments, debugging function, and cross-compiler options removed for  clarity. See \fref{ax:code:netcode} for an example of undoctored code.

\subsection{OS Sockets}

Windows \gls{os} sockets are provided by the \file{<Ws2tcpip.h>} standard \planguage{c} header. Socket references are saved in \code{int} variables, that represent that specific socket's file descriptor.

\subsubsection{Receival Connection}

An in (receiving) socket is represented by a \code{struct sockaddr_in}. The \code{socket()} call returns a file descriptor, and the \code{IPPROTO_UDP} option specifies \gls{udp} as the socket's protocol; additional options are specified with \code{setsockopt()}. Meta properties of the socket (like \acrshort{ip} and port) are directly set into the \code{struct}'s appropriate fields. The socket is set as non-blocking (i.e.\ the execution continues if nothing is found on a \code{recv()} call) by supplying the \code{FIONBIO} option to \code{ioctlsocket()}. Finally, the socket is opened (bound) with \code{bind()}, by passing the file descriptor and the socket's \code{struct}.

\begin{codelist}{UDPConnector.cpp --- open\_RX\_connection()}{73}
bool UDPConnector::open_RX_connection(const char* ip, const uint16_t port) {
	if (m_socketRXfd > -1)
		return false;

	struct sockaddr_in sock;
	uint32_t address;

	int opt;
	int result;

	InetPton(AF_INET, ip, &address);

	m_socketRXfd = socket(AF_INET, SOCK_DGRAM, IPPROTO_UDP);

	if (m_socketRXfd == -1)
		return false;

	opt = 1;
	result = setsockopt(m_socketRXfd, SOL_SOCKET, SO_REUSEADDR, (char*) &opt, sizeof(opt));

	if (result == -1)
		return false;

	memset(&sock, 0, sizeof(sock));
	sock.sin_family      = AF_INET;
	sock.sin_addr.s_addr = address;
	sock.sin_port        = htons(port);

	result = bind(m_socketRXfd, (struct sockaddr*) &sock, sizeof(struct sockaddr_in));

	if (result == -1)
		return false;

	opt = 1;
	ioctlsocket(m_socketRXfd, FIONBIO, (u_long*) &opt);

	return true;
}
\end{codelist}

\subsubsection{Delivery Connection}

Opening an out (delivery) socket is similar, with only minor changes in the options supplied to the socket. The main difference is the usage of \code{connect()} instead of \code{bind()}: the former is used to connect to a remote socket; the latter provides an endpoint for external connections.

\begin{codelist}{UDPConnector.cpp --- open\_TX\_connection()}{143}
bool UDPConnector::open_TX_connection(const char* ip, const uint16_t port) {
	if (m_socketTXfd > -1)
		return false;

	struct sockaddr_in sock;
	uint32_t address;

	int opt;
	int result;

	InetPton(AF_INET, ip, &address);

	m_socketTXfd = socket(AF_INET, SOCK_DGRAM, 0);

	if (m_socketTXfd == -1)
		return false;

	opt = 1;
	result = setsockopt(m_socketTXfd, SOL_SOCKET, SO_REUSEADDR, (char*) &opt, sizeof(opt));

	if (result == -1)
		return false;

	opt = 1;
	result = setsockopt(m_socketTXfd, SOL_SOCKET, SO_BROADCAST, (char*) &opt, sizeof(opt));

	if (result == -1)
		return false;

	memset(&sock, 0, sizeof(sock));
	sock.sin_family      = AF_INET;
	sock.sin_addr.s_addr = address;
	sock.sin_port        = htons(port);

	result = connect(m_socketTXfd, (struct sockaddr*) &sock, sizeof(struct sockaddr_in));

	if (result == -1)
		return false;

	return true;
}
\end{codelist}

\subsubsection{Receiving Packet}

The \code{recv()} function polls the receival socket to see if any \glspl{packet} were sent to it. If one is found its \gls{payload} is saved in \code{m_payloadRX}.

\begin{codelist}{UDPConnector.cpp --- receive\_packet()}{215}
bool UDPConnector::receive_packet() {
	int payloadSize = recv(m_socketRXfd, m_payloadRX, UDPCONNECTOR_ME_PAYLOAD_BUFFER, 0);

	if (payloadSize > 0)
		return true;
	else
		return false;
}
\end{codelist}

\subsubsection{Sending Packet}

The \code{send()} function takes a \gls{payload} (and its size) and sends a \gls{packet} containing it.

\begin{codelist}{UDPConnector.cpp --- send\_packet()}{234}
bool UDPConnector::send_packet() {
	int result = send(m_socketTXfd, m_payloadTX, m_payloadTXSize, 0);

	if (result > 0)
		return true;
	else
		return false;
}
\end{codelist}

\subsection{0MQ Sockets}

\software{\FONTsmallcaps{0mq}} sockets are provided by the \file{<zmq.h>} library \planguage{c} header. Socket references are saved as \code{void} pointers, that represent the memory location of a specific socket. Before creating the sockets a context must be instantiated as well, containing some general configuration settings about \software{\FONTsmallcaps{0mq}}'s process.

The \code{UDPConnector()} constructor instantiates a new context, and specifies the receive and send socket references. The current implementation uses the \FONTsmallcaps{pub/sub} model: a receiving socket subscribes to a remote publishing socket; multiple receiving sockets may be subscribed to a single publish socket at the same time. The project's structure only requires a single subscriber, but the other \software{\FONTsmallcaps{0mq}} model \FONTsmallcaps{req/rep}would force the server and client sendrates to be the same (where each request must be followed by a reply).

\begin{codelist}{UDPConnector.cpp --- UDPConnector()}{63}
UDPConnector::UDPConnector() {
	m_context  = zmq_ctx_new();
	m_socketRX = zmq_socket(m_context, ZMQ_SUB);
	m_socketTX = zmq_socket(m_context, ZMQ_PUB);

	m_payloadRX = (char*) malloc(UDPCONNECTOR_ME_PAYLOAD_BUFFER);
	m_payloadTX = (char*) malloc(UDPCONNECTOR_ME_PAYLOAD_BUFFER);
}
\end{codelist}

\subsubsection{Receival Connection}

Opening a receival connection in \software{\FONTsmallcaps{0mq}} only requires a \code{connect()} call, by specifying the socket reference and the remote address to connect to. \code{zmq_setsockopt()} allows setting further options:

\begin{itemize}
	\item \code{ZMQ_CONFLATE} forces \software{\FONTsmallcaps{0mq}} to only retain the latest \gls{packet} received, and not keep a queue unhandled old \glspl{packet};
	\item \code{ZMQ_SUBSCRIBE} specifies the list the socket subscribes to (in this case, only one unnamed list exists).
\end{itemize}

\begin{codelist}{UDPConnector.cpp --- open\_RX\_connection()}{80}
bool UDPConnector::open_RX_connection(const char* ip, const uint16_t port) {

	std::string addr = std::string(ZMQBRIDGE_REMOTE_PROTOCOL) + "://" + std::string(ip) + ":" + std::to_string(port);

	int conflate = 1;
	zmq_setsockopt(m_socketRX, ZMQ_CONFLATE, &conflate, sizeof(conflate));
	zmq_connect(m_socketRX, addr.c_str());
	zmq_setsockopt(m_socketRX, ZMQ_SUBSCRIBE, "", 0);

	return true;
}
\end{codelist}

\subsubsection{Delivery Connection}

A delivery connection is set up with the \code{bind()} call, that creates a socket to which remote \software{\FONTsmallcaps{0mq}} processes can connect to. The \code{ZMQ_CONFLATE} option is also passed to avoid a \gls{packet} queueing.

\begin{codelist}{UDPConnector.cpp --- open\_TX\_connection()}{95}
bool UDPConnector::open_TX_connection(const char* ip, const uint16_t port) {
	std::string addr = std::string(ZMQBRIDGE_REMOTE_PROTOCOL) + "://" + std::string(ip) + ":" + std::to_string(port);

	int conflate = 1;
	zmq_setsockopt(m_socketTX, ZMQ_CONFLATE, &conflate, sizeof(conflate));
	zmq_bind(m_socketTX, addr.c_str());

	return true;
}
\end{codelist}

\subsubsection{Receiving Packet}

Receiving a \gls{packet} on a socket requires a single \code{zmq_recv()} call, passing the \code{ZMQ_NOBLOCK} option for a non-blocking receive (the execution does not halt if a \gls{packet} is not received). The \gls{payload} is placed in \code{m_payloadRX}.

\begin{codelist}{UDPConnector.cpp --- receive\_packet()}{109}
bool UDPConnector::receive_packet() {
	zmq_recv(m_socketRX, m_payloadRX, UDPCONNECTOR_ME_PAYLOAD_BUFFER, ZMQ_NOBLOCK);

	if (std::string(m_payloadRX).size() > 0)
		return true;
	else
		return false;
}
\end{codelist}

\subsubsection{Sending Packet}

Sending a \gls{packet} requires firstly sending the list name for that \gls{packet} (in this case, an empty string), and then sending the \gls{payload} itself. Both are done with the \code{zmq_send()} function, with the first part specifying \code{ZMQ_SENDMORE} to notify the receiver that more bytes from the same \gls{packet} are to be expected. The \gls{payload} sent is the one found in \code{m_payloadTX}.

\begin{codelist}{UDPConnector.cpp --- send\_packet()}{130}
bool UDPConnector::send_packet() {
	zmq_send(m_socketTX, "", 0, ZMQ_SNDMORE);
	int sentBytes = zmq_send(m_socketTX, m_payloadTX, m_payloadTXSize, 0);

	if (sentBytes > 0)
		return true;
	else
		return false;
}
\end{codelist}

\section{VTD Implementation}\label{sc:middleware:implementation}

For testing purposes and as a proof-of-concept, \gls{vtd} was implemented during the development of the \gls{middleware}.

Following the specs, two functions were overridden:

\begin{itemize}
	\item \code{handle_RX_payload()} converts the data from the \gls{rdb} format into matching \code{TrafficEntity}(ies);
	\item \code{prepare_TX_payload()} converts the data from \code{TrafficEntity}(ies) to the \gls{rdb} format.
\end{itemize}

\gls{vtd} provides a full set of \glspl{api} to create, parse, print, and generally handle data in the \gls{rdb} format. This greatly eases the implementation, since it hides all the details on packaging, memory management, and more. Among the many \gls{rdb} datatypes, the ones relevant to this implementation are \code{RDB_OBJECT_STATE_t} and \code{RDB_VEHICLE_SYSTEMS_t}, that respectively hold data about an object's overall state (dimensions, position, vectors, type, name, etc.), and a vehicle's systems (sensors, lights, etc.).

The higher level methods that convert data are, \gls{rdb} to \gls{middleware}:

\begin{itemize}
	\item \code{convert_entry_to_entity()} --- converts an \gls{rdb} datatype (passed as a memory pointer) into an appropriate \code{TrafficEntity}, adding it to a \code{std::vector<TrafficEntity>};
\end{itemize}

\FLOATnoindent whereas, for \gls{middleware} to \gls{rdb}:

\begin{itemize}
	\item \code{convert_entity_to_objectstate()} --- converts a \code{TrafficEntity} into an appropriate \gls{rdb} \\ \code{RDB_OBJECT_STATE_t}, returning a memory point to that object;
	\item \code{convert_entity_to_vehiclesystems()} --- converts a \code{TrafficEntity} into an appropriate \gls{rdb} \\ \code{RDB_VEHICLE_SYSTEMS_t}, returning a memory point to that object.
\end{itemize}

Since a single \code{rdb} \gls{packet} contains the data for all the entities in the simulation, the method that handles received data must discriminate between the different \gls{rdb} datatypes, additionally keeping count on how many bytes remain before the \gls{packet}'s end. The received \glspl{packet} also contain \code{simTime} and \code{frameNo}, used to temporally synchronize the simulation between \gls{vtd} and \gls{me}.

\begin{codelist}{VTDConnector.cpp --- handle\_RX\_payload()}{65}
std::vector<TrafficEntity*> VTDConnector::handle_RX_payload() {
	std::vector<TrafficEntity*> entities;

	RDB_MSG_t* message = (RDB_MSG_t*) m_payloadRX;

	if (!message || !message->hdr.dataSize || message->hdr.magicNo != RDB_MAGIC_NO)
		return entities;

	RDB_MSG_ENTRY_HDR_t* entry = (RDB_MSG_ENTRY_HDR_t*) (((char*) message) + message->hdr.headerSize);

	m_time  = message->hdr.simTime;
	m_frame = message->hdr.frameNo;

	uint32_t remainingBytes = message->hdr.dataSize;
	while (remainingBytes) {
		int elementsAmount = entry->elementSize ? (entry->dataSize / entry->elementSize) : 0;

		char* dataStart = (char*) entry;
		dataStart += entry->headerSize;

		while (elementsAmount) {
			switch (entry->pkgId) {
				case RDB_PKG_ID_OBJECT_STATE:
					convert_entry_to_entity(&entities, (RDB_OBJECT_STATE_t*) dataStart, entry->flags & RDB_PKG_FLAG_EXTENDED);
					break;
				case RDB_PKG_ID_VEHICLE_SYSTEMS:
					convert_entry_to_entity(&entities, (RDB_VEHICLE_SYSTEMS_t*) dataStart);
					break;
				case RDB_PKG_ID_TRAFFIC_LIGHT:
					convert_entry_to_entity(&entities, (RDB_TRAFFIC_LIGHT_t*) dataStart);
					break;
				default:
					break;
			}
			dataStart += entry->elementSize;
			elementsAmount -= 1;
		}

		remainingBytes -= (entry->headerSize + entry->dataSize);
		if (remainingBytes)
			entry = (RDB_MSG_ENTRY_HDR_t*) (((char*) entry) + entry->headerSize + entry->dataSize);
	}

	return entities;
}
\end{codelist}

Internally, \code{convert_entry_to_entity()} calls another method that checks whether a \code{TrafficEntity} with the found \code{id} is already present in the \code{std::vector}: if so, it gets its reference, and appends the data to it; otherwise, it instantiates a new (appropriate) \code{TrafficEntity} and inserts it into the \code{std::vector}.

\begin{codelist}{VTDConnector.cpp --- prepare\_TX\_payload()}{118}
TrafficEntity* VTDConnector::get_add_entity(std::vector<TrafficEntity*>* entities, const uint8_t id, const TrafficEntity::AVAILABLE_ENTITIES_e type) {
	for (std::vector<TrafficEntity*>::const_iterator iter = entities->begin(); iter != entities->end(); iter++)
		if ((*iter)->id == id)
			return (*iter);

	TrafficEntity* entity;
	switch (type) {
		case TrafficEntity::AVAILABLE_ENTITIES_e::ENTITY_VEHICLE:
			entity = new VehicleEntity();
			break;
		case TrafficEntity::AVAILABLE_ENTITIES_e::ENTITY_TRAFFICLIGHT:
			entity = new LightEntity();
			break;
		case TrafficEntity::AVAILABLE_ENTITIES_e::ENTITY_TRAFFIC:
		default:
			entity = new TrafficEntity();
			break;
	}

	entities->push_back(entity);

	return entity;
}
\end{codelist}

The actual conversion of the raw data is handled vector by vector. A temporary \code{maths::vector vector} is used to apply the transformation matrices, set to transform coordinates from \gls{vtd}'s to the \gls{middleware}'s coordinate system. Firstly, the geometry's center point is moved by calling \code{move_pivot_front()}, that does so depending on the vehicle's yaw (\code{entry->base.pos.h}); the resulting vector is then copied member-by-member to the temporary \code{vector}, and two transform matrices are applied: one to align the scenario, and one to transform the coordinate systems. Finally, the data is copied member-by-member from the temporary \code{vector} to the \code{TrafficEntity}'s \code{inertialPosition} property. A similar process is used for angles, velocity, and acceleration vectors.

\begin{codelist}{VTDConnector.cpp --- convert\_entry\_to\_entity()}{162}
void VTDConnector::convert_entry_to_entity(std::vector<TrafficEntity*>* entities, const RDB_OBJECT_STATE_t* entry, bool isExtended) {
	VehicleEntity* entity = (VehicleEntity*) get_add_entity(entities, entry->base.id - 1, TrafficEntity::AVAILABLE_ENTITIES_e::ENTITY_VEHICLE);

	entity->id = entry->base.id - 1;

	maths::Vector_t vector;

	if (entry->base.pos.type == RDB_COORD_TYPE_INERTIAL) {
		entity->inertialPosition.isValid = true;

		transformMatrices.alpha = entry->base.pos.h;
		transformMatrices.move_pivot_front();

		vector.x = entry->base.pos.x;
		vector.y = entry->base.pos.y;
		vector.z = entry->base.pos.z;

		vector = maths::apply_matrix(&(vector), transformMatrices.testTrack.VTDToME);
		vector = maths::apply_matrix(&(vector), transformMatrices.VTDToSAE);

		entity->inertialPosition.x = vector.x;
		entity->inertialPosition.y = vector.y;
		entity->inertialPosition.z = vector.z;
	}

	entity->orientation.pitch = maths::rad_to_deg(entry->base.pos.p);
	entity->orientation.roll  = maths::rad_to_deg(entry->base.pos.r);
	entity->orientation.yaw   = maths::rad_to_deg(entry->base.pos.h) + 90;

	if (isExtended) {
		vector.x = entry->ext.speed.x;
		vector.y = entry->ext.speed.y;
		vector.z = entry->ext.speed.z;

		vector = maths::apply_matrix(&(vector), transformMatrices.VTDToSAE);

		entity->velocities.x     = vector.x;
		entity->velocities.y     = vector.y;
		entity->velocities.z     = vector.z;
		entity->velocities.pitch = maths::rad_to_deg(entry->ext.speed.p);
		entity->velocities.roll  = maths::rad_to_deg(entry->ext.speed.r);
		entity->velocities.yaw   = maths::rad_to_deg(entry->ext.speed.h) + 90;

		vector.x = entry->ext.accel.x;
		vector.y = entry->ext.accel.y;
		vector.z = entry->ext.accel.z;

		vector = maths::apply_matrix(&(vector), transformMatrices.VTDToSAE);

		entity->accelerations.x     = vector.x;
		entity->accelerations.y     = vector.y;
		entity->accelerations.z     = vector.z;
		entity->accelerations.pitch = maths::rad_to_deg(entry->ext.accel.p);
		entity->accelerations.roll  = maths::rad_to_deg(entry->ext.accel.r);
		entity->accelerations.yaw   = maths::rad_to_deg(entry->ext.accel.h) + 90;
	}

	switch (entry->base.cfgModelId) {
		case 3:
			entity->type = VehicleEntity::TYPE_HATCHBACK;
			break;
		case 5:
		case 15:
			entity->type = VehicleEntity::TYPE_SEDAN;
			break;
	}
}
\end{codelist}

The opposite conversion (\gls{middleware} to \gls{vtd}) is similar.

\section{Testing}\label{sc:middleware:testing}

Testing was performed during and after the development of the \gls{middleware}, by comparing \gls{me}'s visual rendering with \gls{vtd}'s traffic behaviour. The testing environment has \gls{me} running on a Windows 10 machine, whilst \gls{vtd} runs on a guest Linux \gls{vm}: the \gls{vm} is set such that it obtains its own \gls{ip} from the network's \gls{dhcp} server, separating the host and guest and generalizing the environment from an \gls{ip} perspective.

\begin{image}
	{middleware/environment}{1.0}
	{Middleware Development: testing environment}
	{im:middleware:environment}
	{}
	{An overall view of the testing environment, assuming the \gls{middleware} is implemented with \gls{os} sockets. The two host \glspl{os} (one being a \gls{vm}) obtain separate \glspl{ip} from the \gls{dhcp} server, and communicate by specifying the other's \gls{ip} and the correct port for the process.}
\end{image}

Unfortunately, \software{\FONTsmallcaps{0mq}} requires that all the entities part of the system implement their sockets. This becomes a problem when testing the \gls{middleware} with \software{\FONTsmallcaps{0mq}}, since \gls{vtd} is closed source and does not use \software{\FONTsmallcaps{0mq}} socket, nor does it offer the possibility to implement them. To avoid this issue a \gls{proxy} was coded, such that it receives/sends from/to \gls{vtd} on non-\software{\FONTsmallcaps{0mq}} sockets, and forwards the \gls{payload} sending/receiving from/to the \gls{middleware} on \software{\FONTsmallcaps{0mq}} sockets.

\subsection{The Proxy}

\begin{definition}{proxy}
\end{definition}

This \gls{proxy} resides on the \gls{vm} hosting \gls{vtd}; for it to work, the ports are modified on the \gls{middleware} as well, replacing \gls{vtd}'s with the \gls{proxy}'s. This completely hides the \gls{proxy}'s presence to the \gls{middleware}, that has no knowledge that a third software is mediating its communication with \gls{vtd}.\footnote{The additional latency introduced by the \gls{proxy} is unsubstantial, see \fref{sc:performance:proxy} for details.}

\begin{image}
	{middleware/zmqenvironment}{1.0}
	{Middleware Development: \software{\FONTsmallcaps{0mq}} testing environment}
	{im:middleware:zmqenvironment}
	{}
	{This image is similar to \fref{im:middleware:environment}, but with an additional layer (the \gls{proxy}) between the \gls{middleware} and \gls{vtd}. The port numbers also change to reflect this addition.}
\end{image}

The \gls{proxy} opens two connection "lanes", one upstream (data flowing out the \gls{vm}) and one downstream (data flowing in the \gls{vm}). It does so by creating four sockets, two per type: \gls{os} towards \gls{vtd}, \software{\FONTsmallcaps{0mq}} towards the \gls{middleware}. It then waits for \glspl{packet} to be received from either side, and reroutes them without any modifications. The full low level code that handles the rerouting can be found in \fref{ax:code:proxy}.

\begin{codelist}{start.cpp --- main()}{12}
int main() {
	ZMQBridge::Bridge bridge;

	bridge.open_upstream_lane();
	bridge.open_downstream_lane();

	while (true) {
		bridge.flow_upstream();
		bridge.flow_downstream();
	}

	bridge.close_upstream_lane();
	bridge.close_downstream_lane();

	bridge.~Bridge();

	return 0;
}
\end{codelist}

Sockets are created and handled nearly identically as seen in \fref{sc:middleware:netcode}.

\section{Recap}\label{sc:middleware:recap}

The \gls{middleware}'s requirements and matching features were presented and explained in \fref{sc:middleware:features}; its overall structure was then described in \fref{sc:middleware:structure}, by discriminating between helper namespaces, data objects, the core class, and the necessary integration class. \fref{sc:middleware:netcode} summarily details the \gls{netcode} implementation, both with \gls{os} and \software{\FONTsmallcaps{0mq}} sockets. \fref{sc:middleware:implementation} broadly explains how \gls{vtd} was integrated as the project's proof-of-concept. Finally, \fref{sc:middleware:testing} presented the development and testing environment, explaining why a \gls{proxy} is needed to test \software{\FONTsmallcaps{0mq}}, where it resides, and how it works.
