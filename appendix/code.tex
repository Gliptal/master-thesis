\chapter{Code}\label{ax:code}

\section{Unmodified Netcode}\label{ax:code:netcode}

\begin{codelist}{UDPConnector.cpp --- open\_RX\_connection()}{73}
bool UDPConnector::open_RX_connection(const char* ip, const uint16_t port) {
    if (m_socketRXfd > -1)
        return false;

    struct sockaddr_in sock;
    uint32_t address;

    int opt;
    int result;

    InetPton(AF_INET, ip, &address);

#ifdef DEBUG
    logger::log(logger::LOG_TASK, "UDPConnector", __func__, "opening RX connection on %s:%d", ip, port);
#endif

#ifdef _WIN32
    m_socketRXfd = socket(AF_INET, SOCK_DGRAM, IPPROTO_UDP);
#else
    m_socketRXfd = socket(AF_INET, SOCK_DGRAM, 0);
#endif

    if (m_socketRXfd == -1) {
#ifdef DEBUG
        logger::log(logger::LOG_FAIL, "UDPConnector", __func__, "opening RX socket on %s:%d", ip, port);
#endif
        return false;
    }

    opt = 1;
#ifdef _WIN32
    result = setsockopt(m_socketRXfd, SOL_SOCKET, SO_REUSEADDR, (char*) &opt, sizeof(opt));
#else
    int result   = setsockopt(m_socketRXfd, SOL_SOCKET, SO_REUSEADDR, &opt, sizeof(opt));
#endif

    if (result == -1) {
#ifdef DEBUG
        logger::log(logger::LOG_FAIL, "UDPConnector", __func__, "net address cannot be reused");
#endif
        return false;
    }

    memset(&sock, 0, sizeof(sock));
    sock.sin_family      = AF_INET;
    sock.sin_addr.s_addr = address;
    sock.sin_port        = htons(port);

    result = bind(m_socketRXfd, (struct sockaddr*) &sock, sizeof(struct sockaddr_in));

    if (result == -1) {
#ifdef DEBUG
        logger::log(logger::LOG_FAIL, "UDPConnector", __func__, "could not bind socket");
#endif
        return false;
    }

    opt = 1;
#ifdef _WIN32
    ioctlsocket(m_socketRXfd, FIONBIO, (u_long*) &opt);
#else
    ioctl(m_socketRXfd, FIONBIO, sizeof(opt));
#endif

#ifdef DEBUG
    logger::log(logger::LOG_SUCCESS, "UDPConnector", __func__, "RX connection opened on %s:%d", ip, port);
#endif
    return true;
}
\end{codelist}

\section{Proxy Rerouting}\label{ax:code:proxy}

\begin{codelist}{Bridge.cpp --- various}{127}
void Bridge::flow_upstream() {
	std::string payload;

	payload = obtain_local();

	if (payload != "") {
#ifdef TEST
		std::cout << std::left << std::setw(12) << "[ COMM ]" << std::setw(12) << "[ LANE ]" << "UPSTREAM" << std::endl
		<< std::left << std::setw(12) << "[ COMM ]" << std::setw(12) << "[ RECV ]" << payload.size() << std::endl;
#endif

		deliver_remote(payload);

#ifdef TEST
		std::cout << std::left << std::setw(12) << "[ COMM ]" << std::setw(12) << "[ SEND ]" << payload.size() << std::endl;
#endif
	}
}

void Bridge::flow_downstream() {
	std::string payload;

	payload = obtain_remote();

	if (payload != "") {
#ifdef TEST
		std::cout << std::left << std::setw(12) << "[ COMM ]" << std::setw(12) << "[ LANE ]" << "DOWNSTREAM" << std::endl
		<< std::left << std::setw(12) << "[ COMM ]" << std::setw(12) << "[ RECV ]"payload.size() << std::endl;
#endif

		deliver_local(payload);

#ifdef TEST
		std::cout << std::left << std::setw(12) << "[ COMM ]" << std::setw(12) << "[ SEND ]" << payload.size() << std::endl;
#endif
	}
}

std::string Bridge::obtain_local() {
#ifdef PERF
	Evaluator::start();
#endif

	char* payload[ZMQBRIDGE_UDP_BUFFER];

	int payloadSize = recv(m_socketLocalRX, payload, ZMQBRIDGE_UDP_BUFFER, 0);

	if (payloadSize > 0) {
		return std::string((const char*) payload, payloadSize);
	} else {
		return "";
	}
}

std::string Bridge::obtain_remote() {
	std::string payload;

	zmq::help::recv(m_socketRemoteRX);
	payload = zmq::help::recv(m_socketRemoteRX);

	if (payload.size() > 0) {
		return payload;
	} else {
		return "";
	}
}

void Bridge::deliver_local(const std::string payload) {
	send(m_socketLocalTX, payload.c_str(), payload.size(), 0);
}

void Bridge::deliver_remote(const std::string payload) {
	zmq::help::sendmore(m_socketRemoteTX, "");
	zmq::help::send(m_socketRemoteTX, payload);

#ifdef PERF
	Evaluator::end();
	Evaluator::compute_delta();
#endif
}
\end{codelist}

\section{History Matrix}\label{ax:code:historymatrix}

\begin{codelist}{Traffic.hpp --- m\_trafficVehicles}{206}
///
/// @BRIEF  Frame history matrix for traffic vehicles.
///
/// This matrix contains the history of packets for each traffic vehicle. It's statically of
///  n` rows, where `n` is the number of past history kept, and of `m` columns, where `m` is
/// the maximum amount of traffic vehicles that can be simulated. Each column contains the
/// data for the vehicle whose VehicleEntity#id matches the index. The first row is always
/// the current history. Every other cell in the matrix is filled with `nullptr`.
///
/// For example:
///
/// - frame history - 2
/// - maximum traffic - 3
/// - simulated vehicles - 2
///   - 1st vehicle id - 1
///   - 2nd vehicle id - 2
///
/// 1st tick:
/// | 0                | 1                | 2                |
/// | :--------------: | :--------------: | :--------------: |
/// | nullptr          | 1st Vehicle      | 1st Vehicle      |
/// | nullptr          | nullptr          | nullptr          |
///
/// 2nd tick:
/// | 0                | 1                | 2                |
/// | :--------------: | :--------------: | :--------------: |
/// | nullptr          | 2nd Vehicle      | 2nd Vehicle      |
/// | nullptr          | 1st Vehicle      | 1st Vehicle      |
///
/// 3rd tick:
/// | 0                | 1                | 2                |
/// | :--------------: | :--------------: | :--------------: |
/// | nullptr          | 3rd Vehicle      | 3rd Vehicle      |
/// | nullptr          | 2nd Vehicle      | 2nd Vehicle      |
///
/// @NOTE  Since  0  is reserved for player packets, that column is always filled with  nullptr .
///
static Vehicle* m_trafficVehicles[IOTECH_TRAFFIC_DATA_HISTORY][IOTECH_TRAFFIC_MAX_VEHICLES];
\end{codelist}
